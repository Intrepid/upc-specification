\subsection{UPC Atomic Memory Operations {\tt <upc\_amo.h>}}
\label{upc-amo}
\index{\_\_UPC\_AMO\_\_}
\index{upc\_amo.h}

\npf This subsection provides the UPC parallel extensions of [ISO/IEC00 
    Sec 7.19].  All the characteristics of library functions described
    in [ISO/IEC00 Sec 7.1.4] apply to these as well.  Implementations
    that support this interface shall predefine the feature macro {\tt
    \_\_UPC\_AMO\_\_} to the value 1.

% STANDARD HEADERS
\subsubsection{Standard headers}

\np The standard header is

{\tt <upc\_amo.h>}

\np Unless otherwise noted, all of the functions, types and macros specified
    in Section~\ref{upc-amo} are declared by the header {\tt <upc\_amo.h>}.

\np Every inclusion of {\tt <upc\_amo.h>} has the effect of including
    {\tt <upc\_types.h>}.

% COMMON REQUIREMENTS
\subsubsection{Common Requirements}
\label{upc-amo-reqs}
\npf The following requirements apply to all of the functions defined
     in Section~\ref{upc-amo}.

\np The UPC Atomic Memory Operations (AMO) library introduces an
    \emph{atomicity domain}, an object that specifies a single operand type and
    a set of operations over which access to a memory location in a given
    synchronization phase is guaranteed to be atomic if and only if no other
    mechanisms or atomicity domains are used to access the same memory
    location in the same synchronization phase.~%
    \footnote{In particular, this implies that atomicity is only guaranteed
    if AMO operations accessing a given memory location are separated from any 
    other accesses to that location (via direct read/writes or a different domain) 
    by a {\tt upc\_barrier} or {\tt upc\_notify}/{\tt upc\_wait}.}

\np The following table presents the required support for operations and
    operand types

\begin{center}
\begin{tabular}{l|ccccc}
Operand Type   & Accessors & Bit-wise Ops & Numeric Ops \\ \hline
Integer        &  X        &  X           &  X          \\
Floating Point &  X        &              &  X          \\
{\tt UPC\_PTS} &  X        &              &             \\
\end{tabular}
\end{center}

    where
\begin{itemize}
  \item[-] Supported integer types are {\tt UPC\_INT}, {\tt UPC\_UINT},
    {\tt UPC\_LONG}, {\tt UPC\_ULONG}, {\tt UPC\_INT32}, {\tt UPC\_UINT32},
    {\tt UPC\_INT64}, and {\tt UPC\_UINT64}.
  \item[-] Supported floating-point types are {\tt UPC\_FLOAT} and
    {\tt UPC\_DOUBLE}.
  \item[-] Supported accessors are {\tt UPC\_GET}, {\tt UPC\_SET}, and
    {\tt UPC\_CSWAP}.
  \item[-] Supported bit-wise operations are {\tt UPC\_AND}, {\tt UPC\_OR},
    and {\tt UPC\_XOR}.
  \item[-] Supported numeric operations are
    {\tt UPC\_ADD}, {\tt UPC\_SUB}, {\tt UPC\_MULT}, 
    {\tt UPC\_INC}, {\tt UPC\_DEC}, 
    {\tt UPC\_MAX}, and {\tt UPC\_MIN}.
\end{itemize}

\np The value macros listed below are defined in {\tt <upc\_amo.h>}.  
    All other {\tt UPC\_*} value macros used
    in this subsection are defined by {\tt <upc\_types.h>} (see
    \upcopsection{} and \upctypesection{}).

\newcommand\defmacrotab[2]{ & {\tt UPC\_#1}\index{UPC\_#1} & {\tt #2}\\}
\begin{tabular}{ p{30pt} l l }
& Macro name & Specified operation\\
\cline{2-3}
\defmacrotab{GET}       {Read}
\defmacrotab{SET}       {Write or swap}
\defmacrotab{CSWAP}     {Conditional swap}
\defmacrotab{SUB}       {Subtraction}
\defmacrotab{INC}       {Increment by 1}
\defmacrotab{DEC}       {Decrement by 1}
\end{tabular}

% AMO-SPECIFIC TYPES
\subsubsection{AMO-Specific Types}
\label{upc-amo-types}

% -- upc_amodomain_t
\paragraph{The {\tt upc\_amodomain\_t} type}
\index{upc\_amodomain\_t}

\npf The header {\tt <upc\_amo.h>} declares the type
\begin{verbatim}
    upc_amodomain_t
\end{verbatim}

\np The type {\tt upc\_amodomain\_t} is an opaque UPC type.
    {\tt upc\_amodomain\_t} is a shared datatype with incomplete type (as 
    defined in [ISO/IEC00 Sec 6.2.5]).  Objects of type {\tt upc\_amodomain\_t}
    may therefore only be manipulated through pointers.

\np Two pointers that reference the same atomicity domain object will compare
    as equal.  The results of applying {\tt upc\_phaseof()},
    {\tt upc\_threadof()}, and {\tt upc\_addrfield()} to such pointers are
    undefined.

% -- upc_amohint_t
\paragraph{The {\tt upc\_amohint\_t} type}
\index{upc\_amohint\_t}

\npf The header {\tt <upc\_amo.h>} declares the integral type
\begin{verbatim}
    upc_amohint_t
\end{verbatim}

\np The following macros expand to positive integer constant expressions 
    with type {\tt upc\_amohint\_t} and distinct values.
    They allow the specification of a ``hint'' to the library implementation to indicate
    a \emph{preferred} mode of optimization for atomic operations performed on a domain.
\begin{description}
  \item[{\tt UPC\_AMO\_HINT\_DEFAULT == 0}]
    An implementation-defined default mode
  \item[{\tt UPC\_AMO\_HINT\_LATENCY}]
    Favor low-latency atomic memory operations
  \item[{\tt UPC\_AMO\_HINT\_THROUGHPUT}]
    Favor high-throughput atomic memory operations
  \item[{\tt UPC\_AMO\_HINT\_*}]
    Implementation-defined additional hint values
\end{description}

% AMO FUNCTIONS
\newpage
\subsubsection{AMO Functions}
\label{upc-amo-functions}

% -- upc_all_amodomain_alloc()
\paragraph{The {\tt upc\_all\_amodomain\_alloc} function}
\index{upc\_all\_amodomain\_alloc}

{\bf Synopsis}

\npf\vspace{-1.8em}
\begin{verbatim}
    #include <upc_amo.h>
    upc_amodomain_t *upc_all_amodomain_alloc(upc_type_t type,
         upc_op_t ops, upc_amohint_t hints);
\end{verbatim}

{\bf Description}

\np The {\tt upc\_all\_amodomain\_alloc} function dynamically allocates an
    atomicity domain and returns a pointer to it.

\np The {\tt upc\_all\_amodomain\_alloc} function is a {\em collective} function,
    with {\em single-valued} arguments.
    The return value on every thread points to the same atomicity domain
    object.

\np The atomicity domain created supports AMO calls to operate on objects of a
    unique type, specified by the {\tt type} parameter.  The {\tt upc\_type\_t}
    values and the corresponding type they specify are listed in
    \upctypesection{}.

\np The {\tt ops} parameter specifies the atomic operations to be supported by
    the atomicity domain.  The {\tt ops} parameter shall only specify
    operations within the set permitted for {\tt type} (as defined in
    \ref{upc-amo-reqs}), otherwise behavior is undefined.
    Multiple atomic operation value macros from \ref{upc-amo-reqs}
    can be combined by using the bitwise OR operator ($|$), and each value has
    a unique bitwise representation that can be unambiguously tested using the
    bitwise AND operator ({\tt \&}).

\np The implementation is free to ignore the {\tt hints} parameter.

\np EXAMPLE: Collectively allocate an atomicity domain that supports the
    addition, maximum, and minimum operations (i.e., {\tt UPC\_ADD},
    {\tt UPC\_MAX}, {\tt UPC\_MIN}) on signed 64-bit integers (i.e.,
    {\tt int64\_t}).
\begin{verbatim}
    #include <upc_amo.h>
    upc_amodomain_t* domain = upc_all_amodomain_alloc(
         UPC_INT64, UPC_ADD | UPC_MAX | UPC_MIN, 0);
\end{verbatim}
\vfill

% -- upc_all_amodomain_free()
\paragraph{The {\tt upc\_all\_amodomain\_free} function}
\index{upc\_all\_amodomain\_free}

{\bf Synopsis}

\npf\vspace{-1.8em}
\begin{verbatim}
    #include <upc_amo.h>
    void upc_all_amodomain_free(upc_amodomain_t *ptr);
\end{verbatim}

{\bf Description}

\np The {\tt upc\_all\_amodomain\_free} function is a {\em collective} function,
    with the {\em single-valued} argument {\tt ptr}.

\np The {\tt upc\_all\_amodomain\_free} function frees the resources associated with
    the atomicity domain pointed to by {\tt ptr}.  If {\tt ptr} is a null pointer,
    no action occurs.  Otherwise, if the argument does not match a pointer
    earlier returned by the {\tt upc\_all\_amodomain\_alloc} function, or if
    the atomicity domain has been deallocated by a previous call to
    {\tt upc\_all\_amodomain\_free} the behavior is undefined.

\np The atomicity domain referenced by {\tt ptr} is guaranteed to remain valid
    until all threads have entered the call to {\tt upc\_all\_amodomain\_free},
    but the function does not otherwise guarantee any synchronization or
    strict reference.

\np Any subsequent calls to AMO functions from any thread using {\tt ptr} have
    undefined behavior.

% -- upc_amo()
\paragraph{The {\tt upc\_amo\_strict} and {\tt upc\_amo\_relaxed} functions}
\index{upc\_amo\_strict}
\index{upc\_amo\_relaxed}

{\bf Synopsis}

\npf\vspace{-1.8em}
\begin{verbatim}
    #include <upc_amo.h>
    void upc_amo_strict(upc_amodomain_t *domain,
         void * restrict fetch_ptr, upc_op_t op,
         shared void * restrict target,
         const void * restrict operand1,
         const void * restrict operand2);
    void upc_amo_relaxed(upc_amodomain_t *domain,
         void * restrict fetch_ptr, upc_op_t op,
         shared void * restrict target,
         const void * restrict operand1,
         const void * restrict operand2);
\end{verbatim}

{\bf Description}

\np The {\tt op} argument shall specify an operation included in the {\tt ops}
    argument to the {\tt upc\_all\_amodomain\_alloc} call used to construct
    {\tt domain}, otherwise behavior is undefined.

\np The {\tt target} argument shall be a pointer to a location having the type
    specified in the {\tt type} argument to the {\tt upc\_all\_amodomain\_alloc} 
    call used to construct {\tt domain}, otherwise behavior is undefined. 
    The function treats the arguments {\tt fetch\_ptr}, {\tt target}, 
    {\tt operand1} and {\tt operand2} as pointers to this type.

\np The {\tt upc\_amo\_strict} and {\tt upc\_amo\_relaxed} functions perform
    an atomic update of the object pointed to by {\tt target} such that:

    \begin{verse}
      {\tt *target = *target $\oplus$ *operand1},
        where ``$\oplus$'' is the operator specified by the variable {\tt op}
        and {\tt op} $\in$ \{{\tt UPC\_AND}, {\tt UPC\_OR}, {\tt UPC\_XOR},
        {\tt UPC\_ADD}, {\tt UPC\_SUB}, {\tt UPC\_MULT}, {\tt UPC\_MIN}, {\tt UPC\_MAX}\} \\
      {\tt *target = *target + 1}, 
        where {\tt op} is {\tt UPC\_INC} \\
      {\tt *target = *target - 1}, 
        where {\tt op} is {\tt UPC\_DEC} \\
      {\tt *target = (*target == *operand1) ? *operand2 : *target},
        where {\tt op} is {\tt UPC\_CSWAP} \\
      {\tt *target = *operand1},
        where {\tt op} is {\tt UPC\_SET} \\
      {\tt *target} is unchanged,
        where {\tt op} is {\tt UPC\_GET} \\
    \end{verse}

\np The arguments {\tt operand1} and {\tt operand2} shall each be a null
    pointer for those operations that do not require them.%
    \footnote{That is, for all permitted operations other than {\tt UPC\_CSWAP},
    {\tt operand2} shall be a null pointer, and for {\tt UPC\_GET}, {\tt UPC\_INC} and {\tt UPC\_DEC}
    both {\tt operand1} and {\tt operand2} shall be a null pointer.}

\np The value of {\tt *target} prior to performing
    the specified update is stored in {\tt *fetch\_ptr} if and only if
    {\tt fetch\_ptr $\neq$ NULL}.%
    \footnote{If {\tt op} is {\tt UPC\_SET} and {\tt fetch\_ptr} is
    not a null pointer, the effect is an unconditional atomic swap.}
    If {\tt op} is {\tt UPC\_GET}, {\tt fetch\_ptr} shall not be a null pointer.

\np The following requirements apply when {\tt domain} was allocated 
    with {\tt type == UPC\_FLOAT} or {\tt type == UPC\_DOUBLE}:
    If {\tt *target}, {\tt *operand1} or {\tt *operand2} is a {\em signalling NaN} value 
    (as defined in [ISO/IEC00 Sec 5.2.4.2.2]), behavior is undefined.
    If {\tt op} is {\tt UPC\_CSWAP} and {\tt *target} or {\tt *operand1} is a {\em quiet NaN} value 
    (as defined in [ISO/IEC00 Sec 5.2.4.2.2]), behavior is undefined.

\np If {\tt domain} was allocated with {\tt type == UPC\_PTS} and {\tt op} is {\tt UPC\_CSWAP},
    the comparison shall be performed as specified in [UPC Language Specification Sec. 6.4.2];
    specifically, it ignores the phase component of the pointers-to-shared.

\np In all other cases, the value computed by {\tt op} and stored in {\tt *target} 
    shall be equal to the value that would have been computed by passing the operands 
    to the corresponding built-in language operator. In particular, this requires that
    overflows, underflows and quiet NaN values are handled as specified in [ISO/IEC00].

\np The {\tt upc\_amo\_relaxed} function {\em atomically} performs a relaxed shared read of {\tt *target} 
    followed by a relaxed shared write of {\tt *target}. 
    The {\tt upc\_amo\_strict} function {\em atomically} performs a strict shared read of {\tt *target} 
    followed by a strict shared write of {\tt *target}. The write is omitted for {\tt UPC\_GET}, 
    and the read is omitted for {\tt UPC\_SET} when {\tt fetch\_ptr} is a null pointer.
    {\em Atomically} requires the read and write accesses comprising one AMO shall not appear (to any thread) 
    to have been interleaved (or word-torn) with the read/write pair of a conflicting AMO to the same location
    using the same atomicity domain.

\np EXAMPLE: Perform a relaxed atomic fetch-and-increment of a value of type
    {\tt uint64\_t} after allocating an atomicity domain {\tt domain} to
    support {\tt UPC\_INC} for {\tt UPC\_UINT64}.
\begin{verbatim}
    #include <upc_amo.h>
    shared uint64_t val = 42;
    uint64_t oldval;
    upc_amodomain_t* domain = upc_all_amodomain_alloc(
         UPC_UINT64, UPC_INC, 0);
    upc_amo_relaxed(domain, &oldval, UPC_INC, &val, 0, 0);
\end{verbatim}
\vfill

% -- upc_amo_isfast()
\paragraph{The {\tt upc\_amo\_isfast} function}
\index{upc\_amo\_isfast}

{\bf Synopsis}

\npf\vspace{-1.8em}
\begin{verbatim}
    #include <upc_amo.h>
    int upc_amo_isfast(upc_type_t type, upc_op_t ops,
         shared void *addr);
\end{verbatim}

{\bf Description}

\np The {\tt upc\_amo\_isfast} function queries the implementation to determine
    the expected performance of performing a {\tt upc\_amo\_relaxed} call on
    {\tt addr}, using a domain allocated with the arguments {\tt type} and
    {\tt ops}.  The call returns non-zero if the performance is expected to be
    comparable to the fastest expected performance of {\tt upc\_amo\_relaxed}
    for any combination of {\tt addr}, {\tt type}, and {\tt ops}.  Otherwise
    the function returns zero.%
    \footnote{This function allows the implementation to report which
    combinations of type, ops, and alignment are best supported; e.g., using
    hardware atomic instructions.  Some implementations may also return zero
    when {\tt upc\_threadof(addr)} is not equal to the calling thread, to
    indicate the additional cost of remote access.}
