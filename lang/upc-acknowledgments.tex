\section*{Acknowledgments}
                                             
\npf Many have contributed to the ideas and concepts behind
    these specifications.  William Carlson, Jesse Draper,  David Culler, 
Katherine Yelick,
Eugene Brooks, and Karen Warren are the authors of the initial
UPC language concepts and specifications. Tarek El-Ghazawi, 
William Carlson, and Jesse Draper are the authors of the first formal
version of the specifications.  Because of the numerous contributions
to the specifications, no explicit authors are currently mentioned.
We also would like to
acknowledge the role of the participants in the first UPC workshop:
the support and participation of Compaq, Cray, HP, Sun,
and CSC; the abundant input of
Kevin Harris and S\'{e}bastien Chauvin and the efforts of Lauren
Smith; and the efforts of Brian Wibecan and Greg Fischer were
invaluable in bringing these specifications to version 1.0.

\np Version 1.1 is the result of the contributions of many in the UPC
community.  In addition to the
continued support of all those mentioned above, the efforts of Dan
Bonachea were invaluable in this effort.

\np Version 1.2 is also the result of many contributors.  Worthy of special
note (in addition to the
continued support of those mentioned above) are the substantial 
contributions to many aspects of the specifications by Jason Duell;
Many have contributed to the ideas and concepts behind the
UPC collectives specifications.
Elizabeth Wiebel and David Greenberg are the authors of the first draft of
that specification.  Steve Seidel organized the effort to
refine it into its current form.
Thanks go to many in the UPC community for their interest and helpful
comments, particularly Dan Bonachea, Bill Carlson, Jason Duell and
Brian Wibecan.
Version 1.2 also includes the UPC I/O specification which is the result of
efforts by Tarek El Ghazawi, Francois Cantonnet, Proshanta Saha, Rajeev Thakur,
Rob Ross, and Dan Bonachea.
Finally, it also includes the substantial contributions to the UPC memory consistency
model by Kathy Yelick, Dan Bonachea, and Charles Wallace.

\np Members of the UPC consortium may be contacted via the world wide
web at
http://www.upcworld.org or http://upc.gwu.edu, where an archived mailing list
may be joined.  Comments on these specifications are always welcome.
