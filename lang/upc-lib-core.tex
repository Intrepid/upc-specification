\pagebreak
\section{Library}
\subsection{Standard headers}

\npf This subsection provides the UPC parallel extensions of
    [ISO/IEC00 Sec 7.1.2].


\np The standard headers are 

\begin{verbatim}
    <upc_strict.h> 
    <upc_relaxed.h>
    <upc.h> 
\end{verbatim}

\index{upc\_strict.h}
\np Every inclusion of {\tt <upc\_strict.h>} asserts the upc 
    strict pragma and has the effect of including {\tt <upc.h>}.

\index{upc\_relaxed.h}
\np Every inclusion of {\tt <upc\_relaxed.h>} asserts the upc 
    relaxed pragma and has the effect of including {\tt <upc.h>}.

\index{collective}
\np By convention, all UPC standard library functions are  named
    using the prefix {\tt upc\_}.  Those which are collective have
    prefix {\tt upc\_all\_}. 

\pagebreak
\subsection{UPC utilities \tt{<upc.h>}}

\npf This subsection provides the UPC parallel extensions of
    [ISO/IEC00 Sec 7.20].  All of the characteristics of library functions
    described in [ISO/IEC00 Sec 7.1.4] apply to these as well.

\subsubsection{Termination of all threads}
\label{upc_global_exit}

{\bf Synopsis} 

\index{upc\_global\_exit}
\npf\vspace{-2.5em}
\begin{verbatim}
    #include <upc.h> 
     void upc_global_exit(int status);
\end{verbatim}

{\bf Description}

\np{\tt upc\_global\_exit()} flushes all I/O, releases all
      storage, and terminates the execution for all active threads.

\subsubsection{Shared memory allocation functions}
\label{shared_allocation}
\index{shared object, allocation}
\index{memory allocation}

\npf The UPC memory allocation functions return, if successful,
    a pointer-to-shared which is suitably aligned so that it may be assigned 
    to a pointer-to-shared of any type.  The pointer has zero phase and points
    to the start of the allocated space.  If the space cannot be allocated, a null
    pointer-to-shared is returned.
    
\np \xadded[id=DB]{12}{
    There is no required correspondence between the functions specified 
    in Section~\ref{shared_allocation} to allocate and free objects.
    Either of the {\tt upc\_free} or {\tt upc\_all\_free} functions
    may be used to free shared space allocated using {\tt upc\_all\_alloc},
    {\tt upc\_global\_alloc} or {\tt upc\_alloc}.
    }

\paragraph{The {\tt upc\_global\_alloc} function}

{\bf Synopsis} 

\index{upc\_global\_alloc}
\npf\vspace{-2.5em}
\begin{verbatim}
    #include <upc.h> 
    shared void *upc_global_alloc(size_t nblocks, size_t nbytes); 
\end{verbatim}

{\bf Description}

\np The {\tt upc\_global\_alloc} allocates shared space
     compatible with the declaration:

\begin{verbatim}
        shared [nbytes] char[nblocks * nbytes]. 
\end{verbatim}

\np The {\tt upc\_global\_alloc} function is not a {\em
    collective} function. If called by multiple threads,
    all threads which make the call get different allocations.
    If {\tt nblocks*nbytes} is zero, the result is a null pointer-to-shared.
    
\paragraph{The {\tt upc\_all\_alloc} function}

{\bf Synopsis} 
\index{upc\_all\_alloc}

\npf\vspace{-2.5em}
\begin{verbatim}
    #include <upc.h> 
    shared void *upc_all_alloc(size_t nblocks, size_t nbytes); 
\end{verbatim}

{\bf Description}

\np{\tt upc\_all\_alloc} is a {\em collective} function with
    {\em single-valued} arguments.

\np The {\tt upc\_all\_alloc} function allocates  shared 
      space compatible with the following declaration:

\begin{verbatim}
    shared [nbytes] char[nblocks * nbytes].  
\end{verbatim}

\np The {\tt upc\_all\_alloc} function returns the same pointer
   value on all threads. 
   If {\tt nblocks*nbytes} is zero, the result is a null pointer-to-shared.
   
\np The dynamic lifetime of an allocated object extends from
   the time any thread completes the call to {\tt upc\_all\_alloc}
   until any thread has deallocated the object.

\paragraph{The {\tt upc\_alloc} function}

{\bf Synopsis} 
\index{upc\_alloc}

\npf\vspace{-2.5em}
\begin{verbatim}
    #include <upc.h> 
    shared void *upc_alloc(size_t nbytes); 
\end{verbatim}

{\bf Description}

\np The {\tt upc\_alloc} function allocates shared space of at
    least {\tt nbytes} bytes with affinity to the calling thread. 

\np{\tt upc\_alloc} is similar to malloc() except that it
    returns a pointer-to-shared value. It is not a {\em collective} function.
    If {\tt nbytes} is zero, the result is a null pointer-to-shared.

\cbstart
\xdeletedncb[id=DB]{82}{SECTION REMOVED: The {\tt upc\_local\_alloc} function {\em deprecated}}
\cbend

\paragraph{The {\tt upc\_free} function}

{\bf Synopsis} 
\index{upc\_free}

\npf\vspace{-2.5em}
\begin{verbatim}
    #include <upc.h> 
    void upc_free(shared void *ptr); 
\end{verbatim}

{\bf Description}

\np The {\tt upc\_free} function frees the dynamically
    allocated shared storage pointed to by {\tt ptr}.  If {\tt ptr} is
    a null pointer, no action occurs.  Otherwise, if the argument does
    not match a pointer earlier returned by the {\tt upc\_alloc},
\xdeleted[id=DB]{82}{ {\tt upc\_local\_alloc}, }
    {\tt upc\_global\_alloc} or {\tt upc\_all\_alloc} functions,
    or if the space has been deallocated by a previous call
    to {\tt upc\_free} by any thread,\footnote{i.e., only one thread may call {\tt upc\_free} for
     each allocation} 
    \xadded[id=DB]{12}{or a previous call to {\tt upc\_all\_free}}, 
    the behavior is undefined.

\cbstart
\paragraph{The {\tt upc\_all\_free} function}
\xchangenote[id=DB]{12}{SECTION ADDED}

{\bf Synopsis} 
\index{upc\_all\_free}

\npf\vspace{-2.5em}
\begin{verbatim}
    #include <upc.h> 
    void upc_all_free(shared void *ptr); 
\end{verbatim}

{\bf Description}

\np {\tt upc\_all\_free} is a {\em collective} variant of {\tt upc\_free}, provided as a convenience.
    It must be called collectively by all threads with the {\em single-valued} argument {\tt ptr}.

\np The {\tt upc\_all\_free} function frees the dynamically
    allocated shared storage pointed to by {\tt ptr}.  If {\tt ptr} is
    a null pointer, no action occurs.  Otherwise, if the argument does
    not match a pointer earlier returned by the {\tt upc\_alloc},
    {\tt upc\_global\_alloc},  or {\tt upc\_all\_alloc} functions,
    or if the space has been deallocated by a previous call to {\tt upc\_free}
    or {\tt upc\_all\_free}, the behavior is undefined.

\np The shared storage referenced by {\tt ptr} is guaranteed to remain valid
    until all threads have entered the call to {\tt upc\_all\_free}, but the
    function does not otherwise guarantee any synchronization or strict reference.

\cbend

\subsubsection{Pointer-to-shared manipulation functions}

\paragraph{The {\tt upc\_threadof} function}
\label{upc_threadof}

{\bf Synopsis} 
\index{upc\_threadof}
\index{affinity}

\npf\vspace{-2.5em}
\begin{verbatim}
    #include <upc.h> 
    size_t upc_threadof(shared void *ptr); 
\end{verbatim}

{\bf Description}

\np The {\tt upc\_threadof} function returns the index of the
   thread that has affinity to the shared object pointed to by {\tt ptr}.\footnote{%
   This function is used in defining the semantics of pointer-to-shared
   arithmetic in Section \ref{pointer-arithmetic}}

\np If {\tt ptr} is a null pointer-to-shared, the function returns 0.


\paragraph{The {\tt upc\_phaseof} function}
\label{upc_phaseof}

{\bf Synopsis} 
\index{upc\_threadof}
\index{phase}

\npf\vspace{-2.5em}
\begin{verbatim}
    #include <upc.h> 
    size_t upc_phaseof(shared void *ptr); 
\end{verbatim}

{\bf Description}

\np The {\tt upc\_phaseof} function returns the phase component of the
    pointer-to-shared argument.\footnote{%
   This function is used in defining the semantics of pointer-to-shared
   arithmetic in Section \ref{pointer-arithmetic}}

    
\np If {\tt ptr} is a null pointer-to-shared, the function returns 0.

\paragraph{The {\tt upc\_resetphase} function}

{\bf Synopsis} 
\index{upc\_resetphase}
\index{phase}

\npf\vspace{-2.5em}
\begin{verbatim}
    #include <upc.h> 
    shared void *upc_resetphase(shared void *ptr); 
\end{verbatim}

{\bf Description}

\np The {\tt upc\_resetphase} function returns a pointer-to-shared
    which is identical to its input except that it has zero phase. 

\paragraph{The {\tt upc\_addrfield} function}
\label{upc_addrfield}

{\bf Synopsis} 
\index{upc\_addrfield}

\npf\vspace{-2.5em}

\begin{verbatim}
    #include <upc.h> 
    size_t upc_addrfield(shared void *ptr); 
\end{verbatim}

{\bf Description}

\np The {\tt upc\_addrfield} function returns an
   implementation-defined value reflecting the ``local address'' of the
   object pointed to by the pointer-to-shared argument.\footnote{%
   This function is used in defining the semantics of pointer-to-shared
   arithmetic in Section \ref{pointer-arithmetic}}

   
\paragraph{The {\tt upc\_affinitysize} function}

{\bf Synopsis} 
\index{upc\_affinitysize}
\index{upc\_localsizeof}

\npf\vspace{-2.5em}
\begin{verbatim}
    #include <upc.h> 
    size_t upc_affinitysize(size_t totalsize, size_t nbytes, 
         size_t threadid);
\end{verbatim}

{\bf Description}

\np{\tt upc\_affinitysize} is a convenience function which 
    calculates the exact size of the local portion of the data in a 
    shared object with affinity to {\tt threadid}.
    
\np In the case of a dynamically allocated shared object,
    the {\tt totalsize} argument shall be {\tt nbytes*nblocks} and the {\tt nbytes}
    argument shall be {\tt nbytes}, where {\tt nblocks} and {\tt nbytes} are exactly as
    passed to {\tt upc\_global\_alloc} or {\tt upc\_all\_alloc} when the 
    object was allocated.

\np In the case of a statically allocated shared object
    with declaration:

\begin{verbatim}
    shared [b] t d[s];
\end{verbatim}   

    the {\tt totalsize} argument shall be {\tt s * sizeof (t)} and the
    {\tt nbytes} argument shall be {\tt b * sizeof (t)}.  If the block
    size is indefinite, {\tt nbytes} shall be 0.
        
\np{\tt threadid} shall be a value in {\tt 0..(THREADS-1)}.

\subsubsection{Lock functions}
\label{upc_lock}
\index{locks}
\index{synchronization}
\index{mutual exclusion}

\paragraph{Type}

\npf The type declared is 
\index{upc\_lock\_t}

\begin{verbatim}
    upc_lock_t 
\end{verbatim}

\np The type {\tt upc\_lock\_t} is an opaque UPC type. {\tt
    upc\_lock\_t} is a shared datatype with incomplete type (as
    defined in [ISO/IEC00 Sec 6.2.5]).  Objects of type {\tt
    upc\_lock\_t} may therefore only be manipulated through pointers.
    Such objects have two states called {\em locked} and {\em unlocked}.

\np Two pointers to {\tt upc\_lock\_t} that reference the same lock object will
    compare as equal.  The results of applying {\tt upc\_phaseof()},
     {\tt upc\_threadof()}, and {\tt upc\_addrfield()} to such pointers
     are undefined.

\np \xadded[id=DB]{12}{
    There is no required correspondence between the functions specified
    in Section~\ref{upc_lock} to allocate and free locks.
    Either of the {\tt upc\_lock\_free} or {\tt upc\_all\_lock\_free} functions
    may be used to free locks allocated using {\tt upc\_global\_lock\_alloc} or 
    {\tt upc\_all\_lock\_alloc}.
    }

\paragraph{The {\tt upc\_global\_lock\_alloc} function}

{\bf Synopsis} 
\index{upc\_global\_lock\_alloc}

\npf\vspace{-2.5em}
\begin{verbatim}
    #include <upc.h> 
    upc_lock_t *upc_global_lock_alloc(void); 
\end{verbatim}

{\bf Description}

\np The {\tt upc\_global\_lock\_alloc} function dynamically
    allocates a lock and returns a pointer to it.  The lock is created
    in an unlocked state.

\np The {\tt upc\_global\_lock\_alloc} function is not a {\em
    collective} function. If called by multiple threads,
    all threads which make the call get different allocations.
    
\paragraph{The {\tt upc\_all\_lock\_alloc} function}

{\bf Synopsis} 
\index{upc\_all\_lock\_alloc}

\npf\vspace{-2.5em}
\begin{verbatim}
    #include <upc.h> 
    upc_lock_t *upc_all_lock_alloc(void); 
\end{verbatim}

{\bf Description}

\np The {\tt upc\_all\_lock\_alloc} function dynamically
    allocates a lock and returns a pointer to it. The lock is created
    in an unlocked state.

\np The {\tt upc\_all\_lock\_alloc} is a {\em collective}
    function.  The return value on every thread points to the same
    lock object.

\paragraph{The {\tt upc\_lock\_free} function}

{\bf Synopsis} 
\index{upc\_lock\_free}

\npf\vspace{-2.5em}
\begin{verbatim}
    #include <upc.h> 
    void upc_lock_free(upc_lock_t *ptr); 
\end{verbatim}

{\bf Description}

\np The {\tt upc\_lock\_free} function frees all resources
    associated with the dynamically allocated {\tt upc\_lock\_t} pointed to
    by {\tt ptr}.  If {\tt ptr} is a null pointer, no action occurs.
    Otherwise, if the argument does not match a pointer earlier
    returned by the {\tt upc\_global\_lock\_alloc} or {\tt
    upc\_all\_lock\_alloc} function, or if the lock has been
    deallocated by a previous call to {\tt upc\_lock\_free} by any thread,~%
    \footnote
    {i.e., only one thread may call {\tt upc\_lock\_free} for each
    allocation} 
    \xadded[id=DB]{12}{or a previous call to {\tt upc\_all\_lock\_free},}
    the behavior is undefined.

\np{\tt upc\_lock\_free} succeeds regardless of whether the
    referenced lock is currently unlocked or currently locked (by any
    thread).

\np %
\xreplaced[id=DB]{49}{%
    Subsequent or concurrent calls from any thread to functions defined in
    Section~\ref{upc_lock} using the lock referenced by {\tt ptr} have
    undefined behavior.  This also applies to any call to {\tt upc\_lock} on
    the the lock referenced by {\tt ptr} which is blocked at the time of the
    call to {\tt upc\_lock\_free}.
}{%
    Any subsequent calls to locking functions from any
    thread using {\tt ptr} have undefined effects. This also
    applies to any thread currently calling {\tt upc\_lock}.
}

\cbstart
\paragraph{The {\tt upc\_all\_lock\_free} function}
\xchangenote[id=DB]{12}{SECTION ADDED}

{\bf Synopsis}
\index{upc\_all\_lock\_free}

\npf\vspace{-2.5em}
\begin{verbatim}
    #include <upc.h> 
    void upc_all_lock_free(upc_lock_t *ptr); 
\end{verbatim}

{\bf Description}

\np {\tt upc\_all\_lock\_free} is a {\em collective} variant of {\tt upc\_lock\_free}, provided as a convenience.
    It must be called collectively by all threads with the {\em single-valued} argument {\tt ptr}.

\np The {\tt upc\_all\_lock\_free} function frees all resources
    associated with the dynamically allocated {\tt upc\_lock\_t} pointed to
    by {\tt ptr}.  If {\tt ptr} is a null pointer, no action occurs.
    Otherwise, if the argument does not match a pointer earlier
    returned by the {\tt upc\_global\_lock\_alloc} or {\tt
    upc\_all\_lock\_alloc} function, or if the lock has been
    deallocated by a previous call to {\tt upc\_lock\_free}
    or {\tt upc\_all\_lock\_free}, the behavior is undefined.

\np{\tt upc\_all\_lock\_free} succeeds regardless of whether the
    referenced lock is currently unlocked or currently locked (by any
    thread).

\np The lock referenced by {\tt ptr} is guaranteed to remain valid
    until all threads have entered the call to {\tt upc\_all\_lock\_free}, but the
    function does not otherwise guarantee any synchronization or strict reference.

\np Any subsequent calls from any thread to functions defined in
    Section~\ref{upc_lock} using the lock referenced by {\tt ptr} have
    undefined behavior. 

\cbend

\paragraph{The {\tt upc\_lock} function}

{\bf Synopsis} 
\index{upc\_lock}

\npf\vspace{-2.5em}
\begin{verbatim}
    #include <upc.h> 
    void upc_lock(upc_lock_t *ptr); 
\end{verbatim}

{\bf Description}

\np The {\tt upc\_lock} function sets the state of the lock
    pointed to by {\em ptr} to locked.

\np If the lock is already in locked state due to the calling thread setting it
    to locked state, the result is undefined.

\np If the lock is already in locked state, then the calling
    thread waits for some other thread to set the state to 
    unlocked.\footnote{If no other thread calls {\tt upc\_unlock} on {\em ptr}
    the calling thread will never return from this function.}

\np Once the lock is in state unlocked, a single calling thread
    sets the state to locked and the function returns.

\np A null strict access is implied after a call to {\tt
    upc\_lock()}.

\paragraph{The {\tt upc\_lock\_attempt} function}

{\bf Synopsis} 
\index{upc\_lock\_attempt}

\npf\vspace{-2.5em}
\begin{verbatim}
    #include <upc.h> 
    int upc_lock_attempt(upc_lock_t *ptr); 
\end{verbatim}

{\bf Description}

\np The {\tt upc\_lock\_attempt} function attempts to set the state of
   the lock pointed to by {\em ptr} to locked.

\np If the lock is already in locked state due to the calling thread setting it
    to locked state, the result is undefined.

\np If the lock is already in locked state the function returns 0.
    
\np If the lock is in state unlocked, a single calling thread
    sets the state to locked and the function returns 1.

\np A null strict access is implied after a call to {\tt
    upc\_lock\_attempt()} that returns 1.

\paragraph{The {\tt upc\_unlock} function}

{\bf Synopsis} 
\index{upc\_unlock}

\npf\vspace{-2.5em}
\begin{verbatim}
    #include <upc.h> 
    void upc_unlock(upc_lock_t *ptr); 
\end{verbatim}

{\bf Description}

\np The {\tt upc\_unlock} function sets the state of
   the lock pointed to by {\em ptr} to unlocked.

\np Unless the lock is in locked state and the calling
    thread is the locking thread, the result is undefined.

\np A null strict access is implied before a call to {\tt
    upc\_unlock()}.

\subsubsection{Shared string handling functions}
\index{shared object, copying}
\index{memory copy}

\paragraph{The {\tt upc\_memcpy} function}

{\bf Synopsis} 
\index{upc\_memcpy}

\npf\vspace{-2.5em}
\begin{verbatim}
    #include <upc.h> 
    void upc_memcpy(shared void * restrict dst, 
         shared const void * restrict src, size_t n); 
\end{verbatim}

{\bf Description}

\np The {\tt upc\_memcpy} function copies {\tt n} characters
  from a shared object having affinity with one thread to a shared object
  having affinity with the same or another thread.

\np The {\tt upc\_memcpy} function treats the {\tt dst} and
   {\tt src} pointers as if they had type:

\begin{verbatim}
    shared [] char[n] 
\end{verbatim}

   The effect is equivalent to copying the entire contents from one
   shared array object with this type (the {\tt src} array) to another shared
   array object with this type (the {\tt dst} array).

\paragraph{The {\tt upc\_memget} function}

{\bf Synopsis} 
\index{upc\_memget}

\npf\vspace{-2.5em}
\begin{verbatim}
    #include <upc.h> 
    void upc_memget(void * restrict dst, 
         shared const void * restrict src, size_t n); 
\end{verbatim}

{\bf Description}

\np The {\tt upc\_memget} function copies {\tt n} characters from a
   shared object with affinity to any single thread to an object on the calling thread.

\np The {\tt upc\_memget} function treats the {\tt src}
   pointer as if it had type:

\begin{verbatim}
    shared [] char[n] 
\end{verbatim}

   The effect is equivalent to copying the entire contents from one
   shared array object with this type (the {\tt src} array) to an array
   object (the {\tt dst} array) declared with the type

\begin{verbatim}
    char[n] 
\end{verbatim}

\paragraph{The {\tt upc\_memput} function}

{\bf Synopsis} 
\index{upc\_memput}

\npf\vspace{-2.5em}
\begin{verbatim}
    #include <upc.h> 
    void upc_memput(shared void * restrict dst,
         const void * restrict src, size_t n); 
\end{verbatim}

{\bf Description}
 
\np The {\tt upc\_memput} function copies {\tt n} characters from
   an object on the calling thread to a shared object with affinity
   to any single thread.
   
\np The {\tt upc\_memput} function is equivalent to copying the
   entire contents from an array object (the {\tt src} array) declared
   with the type

\begin{verbatim}
    char[n] 
\end{verbatim}

   to a shared array object (the {\tt dst} array) with the type 

\begin{verbatim}
    shared [] char[n] 
\end{verbatim}

\paragraph{The {\tt upc\_memset} function}

{\bf Synopsis}
\index{upc\_memset}
\index{shared object, clearing}

\npf\vspace{-2.5em}
\begin{verbatim}
    #include <upc.h> 
    void upc_memset(shared void *dst, int c, size_t n); 
\end{verbatim}

{\bf Description}

\np The {\tt upc\_memset} function copies the value of {\tt
   c}, converted to an {\tt unsigned char}, to a shared object with affinity
   to any single thread.  The number of bytes set is {\tt n}.

\np The {\tt upc\_memset} function treats the {\tt dst}
   pointer as if had type:

\begin{verbatim}
    shared [] char[n] 
\end{verbatim}

   The effect is equivalent to setting the entire contents of a shared
   array object with this type (the {\tt dst} array) to the value {\tt c}.

