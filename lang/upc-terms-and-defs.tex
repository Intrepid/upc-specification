\addtocontents{toc}{\protect\setcounter{tocdepth}{1}}

% To avoid the introduction of an extra space at the beginning of
% a definition, it is necessary to end each introducing term macro
% invocation below with a %-sign.  This is the LaTeX comment symbol;
% LaTeX will "eat" the newline and subsequent initial white space
% on the next line; this is the desired behavior, necessary to avoid
% the introduction of a spurious white space character.

\section{Terms, definitions and symbols}

\npf For the purpose of these specifications the following
    definitions apply.

\np Other terms are defined where they appear in {\em
    italic} type or on the left hand side of a syntactical rule.

\sterm{thread}% 
     an instance of execution initiated by the execution
     environment at program startup.

\sterm{object}%
     region of data storage in the execution environment
     which can represent values.

\ssterm{shared object}%
    an object allocated using a shared-qualified
    declarator or by a library function defined to create shared objects.
    
\np  NOTE \hspace{5pt}  All threads may access shared 
     objects.\footnote{The file scope declaration
     {\tt shared int x;} creates a single object which any thread may access.}
   
\ssterm{private object}%
     any object which is not a shared object.

\np  NOTE \hspace{5pt} Each thread declares and creates its own 
     private objects which no other thread can access.
     \footnote{The file scope declaration
     {\tt int y;} creates a separate object for each thread to access.}

\ssterm{shared array}%
     an array with elements that have shared qualified type.
  
\sterm{affinity}%
     logical association between shared
     objects and threads.  Each element of data storage that contains
     shared objects has affinity to exactly one thread.
 
\sterm{pointer-to-shared}%
     a pointer whose referenced type is shared-qualified.

\sterm{pointer-to-local}%
     a pointer whose referenced type is not
     shared-qualified.

\label{def-access}
\sterm{access}%
     $<$execution-time action$>$ to read or
     modify the value of an object by a thread.

\ssterm{shared access}%
     an access using an expression whose type
     is shared-qualified.

\pterm{strict shared read}%
     a shared read access which is determined to be
     strict according to section \ref{type_qualifiers} of this specification.

\pterm{strict shared write}%
     a shared modify access which is determined to be
     strict according to section \ref{type_qualifiers} of this specification.

\pterm{relaxed shared read}%
     a shared read access which is determined to be
     relaxed according to section \ref{type_qualifiers} of this specification.

\pterm{relaxed shared write}%
     a shared modify access which is determined to be
     relaxed according to section \ref{type_qualifiers} of this specification.

\ssterm{local access}%
     an access using an expression whose type
     is not shared-qualified.

\index{synchronization}
\sterm{collective}%
     constraint placed on some language 
     operations which requires evaluation of such operations to be
     matched\footnote{A collective operation need not provide any
     actual synchronization between threads, unless otherwise noted.
     The collective requirement simply states a relative ordering property
     of calls to collective operations that must be maintained in the
     parallel execution trace for all executions of any legal program.
     Some implementations may include unspecified synchronization between
     threads within collective operations, but programs must not rely upon
     such unspecified synchronization for correctness.} across all threads.
     The behavior of collective operations is undefined unless all threads
     execute the same sequence of collective operations.

\sterm{single-valued}%
     an operand to a collective operation, which has
     the same value on every thread.  The behavior of the operation
     is otherwise undefined.

\sterm{phase}%
     an unsigned integer value associated with a
     pointer-to-shared which indicates the element-offset within an
     affinity block; used in pointer-to-shared arithmetic
     to determine affinity boundaries.

\addtocontents{toc}{\protect\setcounter{tocdepth}{3}}

\section{Conformance}

\npf In this document, ``shall'' is to be interpreted as a
     requirement on a UPC implementation; conversely, ``shall not'' is to
     be interpreted as a prohibition.

\np  If a ``shall'' or ``shall not'' requirement 
     is violated, the behavior is undefined. Undefined behavior is
     indicated by ``undefined behavior'' or by the omission of any
     explicit definition of behavior from the UPC specification.

\pagebreak
\section{Environment}

\subsection{Conceptual models}
\subsubsection{Translation environment}

\paragraph{Threads environment}
\index{static THREADS environment}
\index{dynamic THREADS environment}
\npf A UPC program is translated under either a {\em static
      THREADS} environment or a {\em dynamic THREADS} environment. Under
      the static THREADS environment, the number of threads to be
      used in execution is indicated to the translator in an
      implementation-defined manner. If the actual execution
      environment differs from this number of threads, the behavior of
      the program is undefined.
       
\subsubsection{Execution environment}

\npf This subsection provides the UPC parallel extensions of
   [ISO/IEC00 Sec. 5.1.2]
   
\index{thread creation}
\np A UPC program consists of a set of threads which may
      allocate both shared and private objects.  
      Accesses to these objects are defined as either
      local or shared, based on the type of the access.  Each thread's local
      accesses behave independently and exactly as described in 
      [ISO/IEC00].  All shared accesses behave as described herein.

\index{implicit barriers}
\np There is an implicit {\tt upc\_barrier} at program startup
     and termination.  Except as explicitly specified by {\tt upc\_barrier} operations
     or by certain library functions (all of which are explicitly documented), there
     are no other barrier synchronization guarantees among the threads.

       {\bf Forward references:} {\tt upc\_barrier} (\ref{upc_barrier}).  

\paragraph{Program startup}

\index{program startup}
\index{main}
\npf In the execution environment of a UPC program, derived
      from the hosted environment as defined in the C Standard [ISO/IEC00],
      each thread calls the UPC program's {\tt main()}
      function\footnote{e.g., in the program {\tt main()\{
      printf("hello"); \} }, each thread prints {\tt hello}.}.

\paragraph{Program termination}

\index{program termination}
\index{upc\_global\_exit}
\index{exit}
\npf A program is terminated by the termination of all the
      threads\footnote{A barrier is automatically inserted at thread termination.} or a call
      to the function {\tt upc\_global\_exit()}.

\np Thread termination follows the C Standard definition of
    program termination in [ISO/IEC00 Sec. 5.1.2.2.3]. A thread is
    terminated by reaching the \}  that terminates the main
    function, by a call to the exit function, or by a return from the
    initial main. Note that thread termination does not imply the
    completion of all I/O and that shared data allocated by a thread
    either statically or dynamically shall not be freed before UPC
    program termination.

     {\bf Forward references:} {\tt upc\_global\_exit} (\ref{upc_global_exit}).  

\paragraph{Program execution}
\label{strict_relaxed}

\index{memory consistency}
\index{shared access}
\index{strict shared read}
\index{strict shared write}
\index{relaxed shared read}
\index{relaxed shared write}
\index{sequential consistency}
\index{program order}
\index{data races}
\npf Thread execution follows the C Standard definition of
  program execution in [ISO/IEC00 Sec. 5.1.2.3].  This section describes
  the additional operational semantics users can expect from accesses to shared
  objects.  In a shared memory model such as UPC, operational
  descriptions of semantics are insufficient to completely
  and definitively describe a memory consistency model.  Therefore
  Appendix \ref{mem-semantics} presents the formal memory
  semantics of UPC.  The information presented in this section is consistent with
  the formal semantic description, but not complete.  Therefore, implementations
  of UPC based on this section alone may be non-compliant.
  
\np All shared accesses are classified as being either strict or relaxed,
  as described in sections \ref{type_qualifiers} and \ref{pragmas}.
  Accesses to shared objects via
  pointers-to-local behave as relaxed shared accesses with respect to memory
  consistency. Most synchronization-related language operations and library
  functions (notably {\it upc\_fence}, {\it upc\_notify}, {\it upc\_wait}, and
  {\it upc\_lock}/{\it upc\_unlock}) imply the consistency effects of a strict
  access.

\np In general, any sequence of purely relaxed shared accesses
  issued by a given
  thread in an execution may appear to be arbitrarily reordered relative to
  program order by the implementation, and different threads need not agree upon
  the order in which such accesses appeared to have taken place. The only
  exception to the previous statement is that two relaxed accesses issued by a
  given thread to the same memory location where at least one is a write will
  always appear to all threads to have executed in program order. Consequently,
  relaxed shared accesses should never be used to perform deterministic
  inter-thread synchronization - synchronization should be performed using
  language/library operations whenever possible, or otherwise using only strict
  shared reads and strict shared writes.

\np Strict accesses always appear (to all threads) to have executed
  in program
  order with respect to other strict accesses, and in a given execution all
  threads observe the effects of strict accesses in a manner consistent with a
  single, global total order over the strict operations. Consequently, an
  execution of a program whose only accesses to shared objects are strict 
  is guaranteed to behave in a sequentially consistent [Lam79] manner.

\np When a thread's program order dictates a set of relaxed
  operations followed by
  a strict operation, all threads will observe the effects of the prior relaxed
  operations made by the issuing thread (in some order) before observing the
  strict operation. Similarly, when a thread's program order dictates a strict
  access followed by a set of relaxed accesses, the strict access will be
  observed by all threads before any of the subsequent relaxed accesses by the
  issuing thread. Consequently, strict operations can be used to synchronize the
  execution of different threads, and to prevent the apparent reordering of
  surrounding relaxed operations across a strict operation.

\np NOTE: It is anticipated that most programs will use the strict 
  synchronization facilities provided by the
  language and library (e.g. barriers, locks, etc) to synchronize threads and
  prevent non-determinism arising from data races. A data race may occur whenever
  two or more relaxed operations from different threads access the same location
  with no intervening strict synchronization, and at least one such access is a
  write. Programs which produce executions that are always free of data races (as
  formally defined in Appendix \ref{mem-semantics}), are guaranteed to behave in
  a sequentially consistent manner.

   {\bf Forward references:} {\tt upc\_fence}, {\tt upc\_notify}, 
    {\tt upc\_wait}, {\tt upc\_barrier} (\ref{upc_barrier}). {\tt upc\_lock},
    {\tt upc\_unlock} (\ref{upc_lock}).
    
