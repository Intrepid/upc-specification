\pagebreak
\section{Proposed Additions and Extensions}
\index{proposed extensions}

\np This section contains proposed additions and extensions to the UPC
    specification.  Such proposals are included when stable enough for
    developers to implement and for users to study and experiment with
    them.  However, their presence does not suggest long term support.
    When fully stable and tested, they will be moved to the main body of
    the specification.

\np This section also describes the process used to add new items to the
    specification, which starts with inclusion in this section.  Requirements
    for inclusion are:\footnote{These requirements ensure that most of
    the semantic issues that arise during initial implementation have
    been addressed and prevents the accumulation of interfaces that no
    one commits to implement. Nothing prevents the circulation of more
    informal {\em what if} interface proposals from circulating in the
    community before an extension reaches this point.}

\begin{enumerate}
\item A documented API which shall use the format and conventions of
    this specification and [ISO/IEC00].

\item Either a complete, publicly available, implementation of the API
    or a set of publicly available example programs which demonstrate
    the interface.
    
\item The concurrence of the UPC consortium that its inclusion would be
    in the best interest of the language.    
\end{enumerate}

\np If all implementations drop support for an extension and/or all interested parties
    no longer believe the extension is worth pursuing, then it may simply be dropped.
    Otherwise, the requirements for inclusion of an extension in the main body of the
    specification are:

\begin{enumerate}
\item Six months residence in this section.

\item The existence of either one (or more) publicly available "reference" implementation 
written in standard UPC OR at least two independent implementations (possibly specific 
to a given UPC implementation).

\item The existence of a significant base of experimental user experience
   which demonstrates positive results with a substantial portion of the
   proposed API.

\item The concurrence of the UPC consortium that its inclusion would be
    in the best interest of the language.
\end{enumerate}

\index{feature macros}
\np For each extension, there shall be a predefined {\em feature macro}
   beginning with {\tt \_\_UPC} which will be defined by an implementation
   to be the interface version of the extension if it is supported, otherwise
   undefined.

