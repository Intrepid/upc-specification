\pagebreak
\section{Library}

\npf This subsection provides the UPC parallel extensions of
    [ISO/IEC00 Sec 7.1.2].
\subsection{UPC Collective Utilities {\tt <upc\_collective.h>}}
\label{upc-collective}
\index{collective libarary}
\index{\_\_UPC\_COLLECTIVE\_\_}
\index{upc\_collective.h}

\npf Implementations that support this interface shall predefine the
    feature macro {\tt \_\_UPC\_COLLECTIVE\_\_} to the value 1.

\np The following requirements apply to all of the functions defined
in this section.

\np All of the functions are collective.

\np All collective function arguments are single-valued.

\np Collective functions may not be called between {\tt upc\_notify}
and the corresponding {\tt upc\_wait}.

\subsubsection{Standard headers}

\np The standard header is

{\tt <upc\_collective.h>}

\subsubsection{Relocalization Operations}

\paragraph{The {\tt upc\_all\_broadcast} function}
\index{upc\_all\_broadcast}
\index{broadcast}

{\bf Synopsis} 

\npf\vspace{-2.5em}
\begin{verbatim}
    #include <upc.h>
    #include <upc_collective.h>
    void upc_all_broadcast(shared void * restrict dst, 
         shared const void * restrict src, size_t nbytes, 
         upc_flag_t flags);
\end{verbatim}

{\bf Description} 

\np The {\tt upc\_all\_broadcast} function copies a block of memory with
affinity to a single thread to a block of shared memory on each thread.
The number of bytes in each block is {\tt nbytes}.

\np {\tt nbytes} must be strictly greater than 0.

\np The {\tt upc\_all\_broadcast} function treats the {\tt src} pointer
as if it pointed to a shared memory area with the type:

\begin{verbatim}
    shared [] char[nbytes]
\end{verbatim}  

\np The effect is equivalent to copying the entire array pointed to by
{\tt src} to each block of {\tt nbytes} bytes of a shared
array {\tt dst} with the type:

\begin{verbatim}
    shared [nbytes] char[nbytes * THREADS]
\end{verbatim}  

\np The target of the {\tt dst} pointer must have affinity to
thread 0.

\np The {\tt dst} pointer is treated as if it has phase 0.

\np EXAMPLE 1 shows {\tt upc\_all\_broadcast}
\begin{verbatim}
  #include <upc.h>
  #include <upc_collective.h>
  shared int A[THREADS];
  shared int B[THREADS];
  // Initialize A.
  upc_barrier;
  upc_all_broadcast( B, &A[1], sizeof(int),
                     UPC_IN_NOSYNC | UPC_OUT_NOSYNC );
  upc_barrier;
\end{verbatim}

\np EXAMPLE 2:
\begin{verbatim}
  #include <upc.h>
  #include <upc_collective.h>
  #define NELEMS 10
  shared [] int A[NELEMS];
  shared [NELEMS] int B[NELEMS*THREADS];
  // Initialize A.
  upc_all_broadcast( B, A, sizeof(int)*NELEMS,
                     UPC_IN_ALLSYNC | UPC_OUT_ALLSYNC );
\end{verbatim}

\np EXAMPLE 3 shows {\tt (A[3],A[4])} is broadcast to
{\tt (B[0],B[1])}, {\tt (B[10],B[11])}, \\
{\tt (B[20],B[21])}, ..., 
{\tt (B[NELEMS*(THREADS-1)],B[NELEMS*(THREADS-1)+1])}.
\begin{verbatim}
  #include <upc.h>
  #include <upc_collective.h>
  #define NELEMS 10
  shared [NELEMS] int A[NELEMS*THREADS];
  shared [NELEMS] int B[NELEMS*THREADS];
  // Initialize A.
  upc_barrier;
  upc_all_broadcast( B, &A[3], sizeof(int)*2,
                     UPC_IN_NOSYNC | UPC_OUT_NOSYNC );
  upc_barrier;
\end{verbatim}

\paragraph{The {\tt upc\_all\_scatter} function}

{\bf Synopsis} 
\index{upc\_all\_scatter}
\index{scatter}

\npf\vspace{-2.5em}
\begin{verbatim}
    #include <upc.h>
    #include <upc_collective.h>
    void upc_all_scatter(shared void * restrict dst, 
         shared const void * restrict src, size_t nbytes, 
         upc_flag_t flags);
\end{verbatim}

{\bf Description} 

\np The {\tt upc\_all\_scatter} function copies the $i$th block of an
area of shared memory with affinity to a single thread
to a block of shared memory with affinity to the $i$th thread.
The number of bytes in each block is {\tt nbytes}.

\np {\tt nbytes} must be strictly greater than 0.

\np The {\tt upc\_all\_scatter} function treats the {\tt src} pointer
as if it pointed to a shared memory area with the type:

\begin{verbatim}
    shared [] char[nbytes * THREADS]
\end{verbatim}  

\np and it treats the {\tt dst} pointer as if it pointed to a shared
memory area with the type:

\begin{verbatim}
    shared [nbytes] char[nbytes * THREADS]
\end{verbatim}  

\np The target of the {\tt dst} pointer must have affinity to thread 0.

\np The {\tt dst} pointer is treated as if it has phase 0.

\np For each thread $i$, the effect is equivalent to copying
the $i$th block of {\tt nbytes} bytes pointed to by {\tt src} to
the block of {\tt nbytes} bytes 
pointed to by {\tt dst} that has affinity to thread $i$.

\np EXAMPLE 1 {\tt upc\_all\_scatter} for the {dynamic THREADS} translation
environment.
%This example corresponds to Figure \ref{fig2}.

\begin{verbatim}
  #include <upc.h>
  #include <upc_collective.h>
  #define NUMELEMS 10
  #define SRC_THREAD 1
  shared int *A;
  shared [] int *myA, *srcA;
  shared [NUMELEMS] int B[NUMELEMS*THREADS];

  // allocate and initialize an array distributed across all threads
  A = upc_all_alloc(THREADS, THREADS*NUMELEMS*sizeof(int));
  myA = (shared [] int *) &A[MYTHREAD];
  for (i=0; i<NUMELEMS*THREADS; i++)
      myA[i] = i + NUMELEMS*THREADS*MYTHREAD;   // (for example)
  // scatter the SRC_THREAD's row of the array
  srcA = (shared [] int *) &A[SRC_THREAD];
  upc_barrier;
  upc_all_scatter( B, srcA, sizeof(int)*NUMELEMS,
                   UPC_IN_NOSYNC | UPC_OUT_NOSYNC);
  upc_barrier;
\end{verbatim}

\np EXAMPLE 2 {\tt upc\_all\_scatter} for the {\em static THREADS} 
translation environment.

\begin{verbatim}
  #include <upc.h>
  #include <upc_collective.h>
  #define NELEMS 10
  shared [] int A[NELEMS*THREADS];
  shared [NELEMS] int B[NELEMS*THREADS];
  // Initialize A.
  upc_all_scatter( B, A, sizeof(int)*NELEMS,
                   UPC_IN_ALLSYNC | UPC_OUT_ALLSYNC );
\end{verbatim}

\paragraph{The {\tt upc\_all\_gather} function}

{\bf Synopsis} 
\index{upc\_all\_gather}
\index{gather}

\npf\vspace{-2.5em} 

\begin{verbatim}
    #include <upc.h>
    #include <upc_collective.h>
    void upc_all_gather(shared void * restrict dst,
         shared const void * restrict src, size_t nbytes,
         upc_flag_t flags);
\end{verbatim}

{\bf Description} 

\np The {\tt upc\_all\_gather} function copies a block of shared memory
that has affinity to the $i$th thread to the $i$th block
of a shared memory area that has affinity to a single thread.
The number of bytes in each block is {\tt nbytes}.

\np {\tt nbytes} must be strictly greater than 0.

\np The {\tt upc\_all\_gather} function treats the {\tt src} pointer
as if it pointed to a shared memory area of {\tt nbytes} bytes on each
thread and therefore had type:

\begin{verbatim}
    shared [nbytes] char[nbytes * THREADS]
\end{verbatim}  

\np and it treats the {\tt dst} pointer as if it pointed to a shared
memory area with the type:

\begin{verbatim}
    shared [] char[nbytes * THREADS]
\end{verbatim} 

\np The target of the {\tt src} pointer must have affinity to thread 0.

\np The {\tt src} pointer is treated as if it has phase 0.

\np For each thread $i$, the effect is equivalent to copying
the block of {\tt nbytes} bytes
pointed to by {\tt src} that has affinity to thread $i$
to the $i$th block of {\tt nbytes} bytes pointed to by {\tt dst}.

\np EXAMPLE 1 {\tt upc\_all\_gather} for the {\em static THREADS}
translation environment.

\begin{verbatim}
  #include <upc.h>
  #include <upc_collective.h>
  #define NELEMS 10
  shared [NELEMS] int A[NELEMS*THREADS];
  shared [] int B[NELEMS*THREADS];
  // Initialize A.
  upc_all_gather( B, A, sizeof(int)*NELEMS,
                  UPC_IN_ALLSYNC | UPC_OUT_ALLSYNC );
\end{verbatim}

\np EXAMPLE 2 {\tt upc\_all\_gather} for the {\em dynamic THREADS}
translation environment.

\begin{verbatim}
  #include <upc.h>
  #include <upc_collective.h>
  #define NELEMS 10
  shared [NELEMS] int A[NELEMS*THREADS];
  shared [] int *B;
  B = (shared [] int *) upc_all_alloc(1,NELEMS*THREADS*sizeof(int));
  // Initialize A.
  upc_barrier;
  upc_all_gather( B, A, sizeof(int)*NELEMS,
                  UPC_IN_NOSYNC | UPC_OUT_NOSYNC );
  upc_barrier;
\end{verbatim}

\paragraph{The {\tt upc\_all\_gather\_all} function}

{\bf Synopsis} 
\index{upc\_all\_gather\_all}
\index{gather, to all}

\npf\vspace{-2.5em} 
\begin{verbatim}
    #include <upc.h>
    #include <upc_collective.h>
    void upc_all_gather_all(shared void * restrict dst,
         shared const void * restrict src, size_t nbytes,
         upc_flag_t flags);
\end{verbatim}

{\bf Description} 

\np
The {\tt upc\_all\_gather\_all} function copies a block of memory from one
shared memory area with affinity to the $i$th thread to the $i$th block
of a shared memory area on each thread.
The number of bytes in each block is {\tt nbytes}.

\np {\tt nbytes} must be strictly greater than 0.

\np The {\tt upc\_all\_gather\_all} function treats the {\tt src} pointer
as if it pointed to a shared memory area of {\tt nbytes} bytes on each
thread and therefore had type:

\begin{verbatim}
    shared [nbytes] char[nbytes * THREADS]
\end{verbatim}  

\np and it treats the {\tt dst} pointer as if it pointed to a shared
memory area with the type:

\begin{verbatim}
    shared [nbytes * THREADS] char[nbytes * THREADS * THREADS]
\end{verbatim} 

\np The targets of the {\tt src} and {\tt dst} pointers
must have affinity to thread 0.

\np The {\tt src} and {\tt dst} pointers are treated as
if they have phase 0.

\np
The effect is equivalent to copying the
$i$th block of {\tt nbytes} bytes pointed to by {\tt src} to the
$i$th block of {\tt nbytes} bytes pointed to by {\tt dst} that
has affinity to each thread.

\np EXAMPLE 1 {\tt upc\_all\_gather\_all} for the {\em static THREADS}
translation environment.

\begin{verbatim}
  #include <upc.h>
  #include <upc_collective.h>
  #define NELEMS 10
  shared [NELEMS] int A[NELEMS*THREADS];
  shared [NELEMS*THREADS] int B[THREADS][NELEMS*THREADS];
  // Initialize A.
  upc_barrier;
  upc_all_gather_all( B, A, sizeof(int)*NELEMS,
                      UPC_IN_NOSYNC | UPC_OUT_NOSYNC );
  upc_barrier;
\end{verbatim}

\np EXAMPLE 2 {\tt upc\_all\_gather\_all} for the {\em dynamic THREADS}
translation environment.

\begin{verbatim}
  #include <upc.h>
  #include <upc_collective.h>
  #define NELEMS 10
  shared [NELEMS] int A[NELEMS*THREADS];
  shared int *Bdata;
  shared [] int *myB;

  Bdata = upc_all_alloc(THREADS*THREADS, NELEMS*sizeof(int));
  myB = (shared [] int *)&Bdata[MYTHREAD];

  // Bdata contains THREADS*THREADS*NELEMS elements.
  // myB is MYTHREAD's row of Bdata.
  // Initialize A.
  upc_all_gather_all( Bdata, A, NELEMS*sizeof(int),
                      UPC_IN_ALLSYNC | UPC_OUT_ALLSYNC );
\end{verbatim}

\paragraph{The {\tt upc\_all\_exchange} function}
\index{upc\_all\_exchange}
\index{exchange}

{\bf Synopsis} 

\npf\vspace{-2.5em}
\begin{verbatim}
    #include <upc.h>
    #include <upc_collective.h>
    void upc_all_exchange(shared void * restrict dst, 
         shared const void * restrict src, size_t nbytes,
         upc_flag_t flags);
\end{verbatim}

{\bf Description} 

\np The {\tt upc\_all\_exchange} function copies the $i$th block of memory
from a shared memory area that has affinity to thread $j$ to the $j$th block
of a shared memory area that has affinity to thread $i$.
The number of bytes in each block is {\tt nbytes}.

\np {\tt nbytes} must be strictly greater than 0.

\np The {\tt upc\_all\_exchange} function treats the {\tt src} pointer
and the {\tt dst} pointer as if each
pointed to a shared memory area of {\tt nbytes}$*${\tt THREADS} bytes
on each thread and therefore had type:

\begin{verbatim}
    shared [nbytes * THREADS] char[nbytes * THREADS * THREADS]
\end{verbatim}  

\np The targets of the {\tt src} and {\tt dst} pointers
must have affinity to thread 0.

\np The {\tt src} and {\tt dst} pointers are treated as
if they have phase 0.

\np For each pair of threads $i$ and $j$, the effect is equivalent to copying
the $i$th block of {\tt nbytes} bytes that has affinity to thread $j$
pointed to by {\tt src}
to
the $j$th block of {\tt nbytes} bytes that has affinity to thread $i$ 
pointed to by {\tt dst}.

\np EXAMPLE 1 {\tt upc\_all\_exchange} for the {\em static THREADS}
translation environment.

\begin{verbatim}
  #include <upc.h>
  #include <upc_collective.h>
  #define NELEMS 10
  shared [NELEMS*THREADS] int A[THREADS][NELEMS*THREADS];
  shared [NELEMS*THREADS] int B[THREADS][NELEMS*THREADS];
  // Initialize A.
  upc_barrier;
  upc_all_exchange( B, A, NELEMS*sizeof(int),
                    UPC_IN_NOSYNC | UPC_OUT_NOSYNC );
  upc_barrier;
\end{verbatim}

\np EXAMPLE 2 {\tt upc\_all\_exchange} for the {\em dynamic THREADS}
translation environment.

\begin{verbatim}
  #include <upc.h>
  #include <upc_collective.h>
  #define NELEMS 10
  shared int *Adata, *Bdata;
  shared [] int *myA, *myB;
  int i;

  Adata = upc_all_alloc(THREADS*THREADS, NELEMS*sizeof(int));
  myA = (shared [] int *)&Adata[MYTHREAD];
  Bdata = upc_all_alloc(THREADS*THREADS, NELEMS*sizeof(int));
  myB = (shared [] int *)&Bdata[MYTHREAD];

  // Adata and Bdata contain THREADS*THREADS*NELEMS elements.
  // myA and myB are MYTHREAD's rows of Adata and Bdata, resp.

  // Initialize MYTHREAD's row of A.  For example,
  for (i=0; i<NELEMS*THREADS; i++)
      myA[i] = MYTHREAD*10 + i;

  upc_all_exchange( Bdata, Adata, NELEMS*sizeof(int),
                    UPC_IN_ALLSYNC | UPC_OUT_ALLSYNC );
\end{verbatim}

\paragraph{The {\tt upc\_all\_permute} function}
\index{upc\_all\_permute}
\index{permute}

{\bf Synopsis} 

\npf\vspace{-2.5em}
\begin{verbatim}
    #include <upc.h>
    #include <upc_collective.h>
    void upc_all_permute(shared void * restrict dst,
        shared const void * restrict src, 
        shared const int * restrict perm,
        size_t nbytes, upc_flag_t flags);
\end{verbatim}

{\bf Description} 

\np The {\tt upc\_all\_permute} function copies a block of memory from a
shared memory area that has affinity to the $i$th thread to a block of a
shared memory that has affinity to thread {\tt perm[i]}.
The number of bytes in each block is {\tt nbytes}.

\np {\tt nbytes} must be strictly greater than 0.

\np {\tt perm[0..THREADS-1]} must contain {\tt THREADS} distinct
values: {\tt 0, 1, ...,  THREADS-1}.

\np The {\tt upc\_all\_permute} function treats the {\tt src} pointer
and the {\tt dst} pointer as if each pointed to a shared memory
area of {\tt nbytes} bytes on each thread and therefore had type:

\begin{verbatim}
    shared [nbytes] char[nbytes * THREADS]
\end{verbatim}  

\np The targets of the {\tt src}, {\tt perm}, and
{\tt dst} pointers must have affinity to thread 0.

\np The {\tt src} and {\tt dst} pointers are treated as
if they have phase 0.

\np The effect is equivalent to copying the block of {\tt nbytes} bytes
that has affinity to thread {\tt i} pointed to by {\tt src}
to the block of {\tt nbytes} bytes
that has affinity to thread {\tt perm}[$i$] pointed to by {\tt dst}.

\np EXAMPLE 1 {\tt upc\_all\_permute}.
\begin{verbatim}
  #include <upc.h>
  #include <upc_collective.h>
  #define NELEMS 10
  shared [NELEMS] int A[NELEMS*THREADS], B[NELEMS*THREADS];
  shared int P[THREADS];
  // Initialize A and P.
  upc_barrier;
  upc_all_permute( B, A, P, sizeof(int)*NELEMS,
                   UPC_IN_NOSYNC | UPC_OUT_NOSYNC );
  upc_barrier;
\end{verbatim}

\subsubsection{Computational Operations}
\label{upc-op-t-section}
\index{upc\_op\_t}
\index{UPC\_ADD}
\index{UPC\_MULT}
\index{UPC\_AND}
\index{UPC\_OR}
\index{UPC\_XOR}
\index{UPC\_LOGAND}
\index{UPC\_LOGOR}
\index{UPC\_MIN}
\index{UPC\_MAX}
\index{UPC\_FUNC}
\index{UPC\_NONCOMM\_FUNC}

\npf
\label{upc-op-t-item}
A variable of type {\tt upc\_op\_t} can have the following values:
\begin{description}
\item[{\tt UPC\_ADD}]
Addition.
\item[{\tt UPC\_MULT}]
Multiplication.
\item[{\tt UPC\_AND}]
Bitwise {\tt AND} for integer and character variables.
Results are undefined for floating point numbers.
\item[{\tt UPC\_OR}]
Bitwise {\tt OR} for integer and character variables.
Results are undefined for floating point numbers.
\item[{\tt UPC\_XOR}]
Bitwise {\tt XOR} for integer and character variables.
Results are undefined for floating point numbers.
\item[{\tt UPC\_LOGAND}]
Logical {\tt AND} for all variable types.
\item[{\tt UPC\_LOGOR}]
Logical {\tt OR} for all variable types.
\item[{\tt UPC\_MIN}]
For all data types, find the minimum value.
\item[{\tt UPC\_MAX}]
For all data types, find the maximum value.
\item[{\tt UPC\_FUNC}]
Use the specified commutative function {\tt func} to operate
on the data in the {\tt src} array at each step.
\item[{\tt UPC\_NONCOMM\_FUNC}]
Use the specified non-commutative function {\tt func} to
operate on the data in the {\tt src} array at each step.
\end{description}

\np The operations represented by a variable of type {\tt upc\_op\_t}
(including user-provided operators) are assumed to be associative.
A reduction or a prefix reduction whose result is dependent on the
order of operator evaluation will have undefined results.\footnote{
Implementations are not obligated to prevent failures that
might arise because of a lack of associativity of built-in functions
due to floating-point roundoff or overflow.}

\np The operations represented by a variable of type {\tt upc\_op\_t}
(except those provided using {\tt UPC\_NONCOMM\_FUNC}) are assumed
to be commutative.  A reduction or a prefix reduction (using operators
other than {\tt UPC\_NONCOMM\_FUNC}) whose result is dependent on
the order of the operands will have undefined results.

{\bf Forward references:} reduction, prefix reduction (\ref{reduction}).

\paragraph{The {\tt upc\_all\_reduce} and {\tt upc\_all\_prefix\_reduce} functions}
\label{reduction}

{\bf Synopsis} 
\index{upc\_all\_reduce}
\index{upc\_all\_reduce\_prefix}
\index{reduction}
\index{prefix reduction}

\npf 
\begin{verbatim}
#include <upc.h>
#include <upc_collective.h>
void upc_all_reduce_<<T>>(
        shared void * restrict dst,
        void shared const void * restrict src,
	upc_op_t op,
	size_t nelems,
        void size_t blk_size,
	<<TYPE>>(*func)(<<TYPE>>, <<TYPE>>),
        void upc_flag_t flags);
void upc_all_prefix_reduce<<T>>(
        shared void * restrict dst,
        void  shared const void * restrict src,
	upc_op_t op,
	size_t nelems,
        void size_t blk_size,
	<<TYPE>>(*func)(<<TYPE>>, <<TYPE>>),
        void upc_flag_t flags);
\end{verbatim}
        
{\bf Description} 

\np The function prototypes above represents the 22 variations of the
  {\tt upc\_all\_reduce{\em T}} and {\tt upc\_all\_prefix\_reduce{\em T}} 
  functions where {\tt {\em T}} and {\tt {\em TYPE}} have the following 
correspondences: \footnote{For example, if {\tt {\em T}} is {\tt C}, then 
{\tt {\em TYPE}} must be {\tt signed char}.}
\begin{center}
\begin{tabular}{ll|ll}
{\tt {\em T}} & {\tt {\em TYPE}} \hspace*{1.5in} &
{\tt {\em T}} & {\tt {\em TYPE}} \\ \hline
{\tt C} & {\tt signed char} &
{\tt L} & {\tt signed long} \\
{\tt UC} & {\tt unsigned char} &
{\tt UL} & {\tt unsigned long} \\
{\tt S} & {\tt signed short} &
{\tt F} & {\tt float} \\
{\tt US} & {\tt unsigned short} &
{\tt D} & {\tt double} \\
{\tt I} & {\tt signed int} &
{\tt LD} & {\tt long double} \\
{\tt UI} & {\tt unsigned int} &
\end{tabular}
\end{center}

\np On completion of the {\tt upc\_all\_reduce} variants, 
the value of the {\tt {\em TYPE}} shared object
referenced by {\tt dst} is
{\tt src[0]} $\oplus$ {\tt src[1]} $\oplus \cdots \oplus$
{\tt src[nelems-1]}
where ``$\oplus$'' is the operator specified by the variable {\tt op}.

\np On completion of the {\tt upc\_all\_prefix\_reduce} variants, 
the value of the {\tt {\em TYPE}} shared object
referenced by {\tt dst[i]} is
{\tt src[0]} $\oplus$ {\tt src[1]}
$\oplus \cdots \oplus$ {\tt src[i]}
for $0 \leq$ {\tt i} $\leq$ {\tt nelems-1} and
where ``$\oplus$'' is the operator specified by the variable {\tt op}.

\np
If the value of {\tt blk\_size} passed to these functions is
greater than 0 then they treat the {\tt src} pointer
as if it pointed to a shared memory area of {\tt nelems} elements of
type {\tt {\em TYPE}} and blocking factor {\tt blk\_size}, and therefore
had type:

\begin{verbatim}
    shared [blk_size] TYPE [nelems]
\end{verbatim}

\np
If the value of {\tt blk\_size} passed to these functions is
0 then they treat the {\tt src} pointer
as if it pointed to a shared memory area of {\tt nelems} elements of
type {\tt {\em TYPE}} with an indefinite layout qualifier, and
therefore had
type\footnote{Note that {\tt upc\_blocksize(src) == 0} if
{\tt src} has this type, so the argument value 0 has a natural
connection to the block size of {\tt src}.}:

\begin{verbatim}
    shared []  TYPE[nelems]
\end{verbatim}

\np The phase of the {\tt src} pointer is respected when
referencing array elements, as specified above.

\np {\tt upc\_all\_prefix\_reduce{\em T}} treats the {\tt dst} pointer
    equivalently to the {\tt src} pointer as described in the past 3
    paragraphs.
    
\np {\tt upc\_all\_prefix\_reduce{\em T}} requires the affinity and
phase of the {\tt src} and {\tt dst} pointers to match -- ie. 
{\tt upc\_threadof(src) == upc\_threadof(dst) \&\& upc\_phaseof(src) == upc\_phaseof(dst)}.

\np {\tt upc\_all\_reduce{\em T}} treats the {\tt dst} pointer as having type:

\begin{verbatim}
    shared TYPE *
\end{verbatim}

\np EXAMPLE 1 {\tt upc\_all\_reduce} of type {\tt long UPC\_ADD}.
\begin{verbatim}
  #include <upc.h>
  #include <upc_collective.h>
  #define BLK_SIZE 3
  #define NELEMS 10
  shared [BLK_SIZE] long A[NELEMS*THREADS];
  shared long *B;
  long result;
  // Initialize A.  The result below is defined only on thread 0.
  upc_barrier;
  upc_all_reduceL( B, A, UPC_ADD, NELEMS*THREADS, BLK_SIZE,
                   NULL, UPC_IN_NOSYNC | UPC_OUT_NOSYNC );
  upc_barrier;
\end{verbatim}

%\paragraph{The {\tt upc\_all\_prefix\_reduce} function}
%\label{prefix-reduction}
%
% {\bf Synopsis} 
%
%1\hspace{1em}   {\tt \#include <upc.h>}\\
%        {\tt \#include <upc\_collective.h>}\\
%        {\tt void {upc\_all\_prefix\_reduce}{\em T}(restrict shared void *dst, \\
%         \phantom{void {upc\_all\_prefix\_reduce}{\em T}(}restrict shared const void *src,\\
%\phantom{void {upc\_all\_prefix\_reduce}{\em T}(}upc\_op\_t op, size\_t nelems, size\_t blk\_size,\\
%\phantom{void {upc\_all\_prefix\_reduce}{\em T}(}{\em TYPE} (*func)({\em TYPE}, {\em TYPE}), \\
%\phantom{void {upc\_all\_prefix\_reduce}{\em T}(}upc\_flag\_t flags);} \\
%
% {\bf Description} 
%
%1\hspace{1em} The function prototype above represents the 11 variations of the
%{\tt upc\_all\_reduce{\em T}} function where {\tt {\em T}} and {\tt {\em TYPE}} have the
%following correspondences:
%
%\begin{center}
%\begin{tabular}{ll|ll}
%{\tt {\em T}} & {\tt {\em TYPE}} \hspace*{1.5in} &
%{\tt {\em T}} & {\tt {\em TYPE}} \\ \hline
%{\tt C} & {\tt signed char} &
%{\tt L} & {\tt signed long} \\
%{\tt UC} & {\tt unsigned char} &
%{\tt UL} & {\tt unsigned long} \\
%{\tt S} & {\tt signed short} &
%{\tt F} & {\tt float} \\
%{\tt US} & {\tt unsigned short} &
%{\tt D} & {\tt double} \\
%{\tt I} & {\tt signed int} &
%{\tt LD} & {\tt long double} \\
%{\tt UI} & {\tt unsigned int} &
%\end{tabular}
%\end{center}
%
%2\hspace{1em} For example, if {\tt {\em T}} is {\tt C}, then {\tt {\em TYPE}} must be {\tt signed char}.
%
%3\hspace{1em}
%\label{prefix-reduce-blksize-gt-0}
%If the value of {\tt blk\_size} passed to {\tt upc\_all\_prefix\_reduce{\em T}} is
%greater than 0 then
%{\tt upc\_all\_prefix\_reduce{\em T}} treats the {\tt src} pointer and the
%{\tt dst} pointer
%as if each pointed to a shared memory area of {\tt nelems} elements of
%type {\tt {\em TYPE}} and blocking factor {\tt blk\_size}, and therefore
%had type:
%
%\begin{verbatim}
%    shared [blk_size}] TYPE[nelems]
%\end{verbatim}
%
%4\hspace{1em}
%\label{prefix-reduce-blksize-0}
%If the value of {\tt blk\_size} passed to {\tt upc\_all\_prefix\_reduce{\em T}} is
%0 then \\
%{\tt upc\_all\_prefix\_reduce{\em T}} treats the {\tt src} pointer and the
%{\tt dst} pointer
%as if each pointed to a shared memory area of {\tt nelems} elements of
%type {\tt {\em TYPE}} with an indefinite layout qualifier, and
%therefore had
%type\footnote{Note that {\tt upc\_blocksize(src) == 0} if
%{\tt src} has this type, so the argument value 0 has a natural
%connection to the block size of {\tt src}.}:
%
%\begin{verbatim}
%    shared [] TYPE [nelems]
%\end{verbatim}
%
%% 1\hspace{1em} The targets of the {\tt src} and {\tt dst} pointers
%% must have affinity to thread 0.
%
%5\hspace{1em} The phases of the {\tt src} and {\tt dst} pointers are
%respected when referencing array elements, as specified in items
%\ref{prefix-reduce-blksize-gt-0} and \ref{prefix-reduce-blksize-0} above.
%
%6\hspace{1em} At function exit
%{\tt dst[i]} = {\tt src[0]} $\oplus$ {\tt src[1]}
%$\oplus \cdots \oplus$ {\tt src[i]}
%for $0 \leq$ {\tt i} $\leq$ {\tt nelems-1} and
%where ``$\oplus$'' is the operator specified by the variable {\tt op}.
%
\np EXAMPLE 2 {\tt upc\_all\_prefix\_reduce} of type {\tt long UPC\_ADD}.

\begin{verbatim}
  #include <upc.h>
  #include <upc_collective.h>
  #define BLK_SIZE 3
  #define NELEMS 10
  shared [BLK_SIZE] long A[NELEMS*THREADS];
  shared [BLK_SIZE] long B[NELEMS*THREADS];
  // Initialize A.
  upc_all_prefix_reduceL( B, A, UPC_ADD, NELEMS*THREADS, BLK_SIZE,
                          NULL, UPC_IN_ALLSYNC | UPC_OUT_ALLSYNC );
\end{verbatim}

\pagebreak
\appendix
\section{Proposed Additions and Extensions}
\index{proposed extensions}

\np This section contains proposed additions and extensions to the UPC
     specification.  Such proposals are included when stable enough for
     developers to implement and for users to study and experiment with them.
     However, their presence does not suggest long term support.  When fully
     stable and tested, they will be moved to the main body of the specification.

\np This section also describes the process used to add new items to the 
    specification, which starts with inclusion in this section.  Requirements
    for inclusion are:\footnote{These requirements ensure that most of the
    semantic issues that arise during initial implementation have been addressed
    and prevents the accumulation of interfaces that no one commits to
    implement. Nothing prevents the circulation of more informal {\em what if} 
    interface proposals from circulating in the community before an extension
    reaches this point.} 

\begin{enumerate}
\item A documented API which shall use the format and conventions of
    this specification and [ISO/IEC00].

\item Either a complete, publicly available, implementation of the API
    or a set of publicly available example programs which demonstrate
    the interface.
    
\item The concurrence of the UPC consortium that its inclusion would be
    in the best interest of the language.    
\end{enumerate}

\np If all implementations drop support for an extension and/or all interested parties
    no longer believe the extension is worth pursuing, then it may simply be dropped.
    Otherwise, the requirements for inclusion of an extension in the main body of the
    specification are:

\begin{enumerate}
\item Six months residence in this section.

\item The existence of either one (or more) publicly available "reference" implementation 
written in standard UPC OR at least two independent implementations (possibly specific 
to a given UPC implementation).

\item The existence of a significant base of experimental user experience
   which demonstrates positive results with a substantial portion of the
   proposed API.

\item The concurrence of the UPC consortium that its inclusion would be
    in the best interest of the language.
\end{enumerate}

\index{feature macros}
\np For each extension, there shall be a predefined {\em feature macro}
   beginning with {\tt \_\_UPC} which will be defined by an implementation
   to be the interface version of the extension if it is supported, otherwise
   undefined.

