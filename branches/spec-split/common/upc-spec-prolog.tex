\documentclass[12pt,titlepage, 
               bookmarks,bookmarksopen=false,
               bookmarksnumbered=true,pdfpagemode=UseOutlines,
               colorlinks=true]{article}

\usepackage[noindentafter]{titlesec}
\usepackage[all]{xy}
\usepackage{html}

% The special comments below are processed by latex2html.
% Everything between them will be skipped by latex2html.
%begin{latexonly}
  \usepackage{ifpdf}
  \ifpdf
    \usepackage{pdfsync}
    \usepackage{hyperref}
  \fi
%end{latexonly}
\usepackage{makeidx}

\setcounter{secnumdepth}{4}
\setcounter{tocdepth}{3}
\hypersetup{bookmarksdepth=4}

% Make paragraphs behave like subsubsections
% See: http://www.charlietanksley.net/philtex/\
%      modifying-section-commands-with-the-titlesec-package/
\titleformat{\paragraph}[hang]{\bfseries}{\theparagraph}{1em}{}
\titlespacing*{\paragraph}{0pt}{1.25ex plus 0.5ex minus 0.25 ex}{0.5ex}

\newcounter{parnum}
\makeatletter
\@addtoreset{parnum}{section}
\@addtoreset{parnum}{subsection}
\@addtoreset{parnum}{subsubsection}
\@addtoreset{parnum}{paragraph}
\makeatother

% Not really a tab, but something like it for HTML
% output.  If we want real tabs, we'll probably need to use CSS.
\newcommand{\tab}{\texttt{~~~~~~}}

\newcommand{\np}{%
\addtocounter{parnum}{1}%
\latex{\hspace{-2em}\makebox[2em][l]{\arabic{parnum}}}%
\html{{\bf {\arabic{parnum}}}\tab}}

\newcommand{\npf}{\setcounter{parnum}{0}\np}

\newcommand{\sterm}[1]{\subsection{\html{#1}}
\npf \latex{{\bf #1}\\}
\index{#1}}

\newcommand{\ssterm}[1]{\subsubsection{\html{#1}}
\npf \latex{{\bf #1}\\}
\index{#1}}

\newcommand{\pterm}[1]{\paragraph{\html{#1}}
\npf \latex{{\bf #1}\\}
\index{#1}}

\setlength{\parindent}{0pt}
\setlength {\parskip}{1.3ex}
