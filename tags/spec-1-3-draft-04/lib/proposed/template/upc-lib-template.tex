\subsection{Template Functions \texttt{<}upc\_template.h\texttt{>}}
\label{upc-template}
\index{upc\_template.h}
\index{\_\_UPC\_TEMPLATE\_\_}

\npf This is a Latex template for writing UPC library proposals. Instructions:
\begin{enumerate}
\item Choose a "slug" name for your library (e.g. "castability")
\item Copy the directory containing this file and all its contents to a new subdirectory of proposed, named by the slug
\item Rename all the files in that directory, replacing "template" with the slug
\item Search and replace the string "template" in each file with the slug
\item Add new paragraphs that follow the skeleton below for each proposed function
\item Replace this text with general introductory material for your library.
\end{enumerate}

\np Implementations that support this interface shall predefine the
feature macro {\_\_UPC\_TEMPLATE\_\_} to the value 1.

\subsubsection{Standard headers}

\npf The standard header is

{\tt <upc\_template.h>}

\np Unless otherwise noted, all of the functions, types and macros
specified in Section~\ref{upc-template}
are declared by the header {\tt <upc\_template.h>}.

\subsubsection{Template Functions}

\paragraph{The {\tt upc\_template} function}
\index{upc\_template}

{\bf Synopsis}

\npf\vspace{-2.5em}
\begin{verbatim}
    #include <upc_template.h>
    void *upc_template(const shared void *ptr);
\end{verbatim}

{\bf Description}

\np The {\tt upc\_template} function does something interesting.
Replace this text with what it does.


