% By default create final copy
\providecommand{\draftorfinal}{\setboolean{draft}{false}}
% Set default document options for 'final' output.
\providecommand{\docoptions}{
   \documentclass[
       12pt,
       titlepage,
       twoside,
       bookmarks,
       bookmarksopen=false,
       bookmarksnumbered=true,
       pdfpagemode=UseOutlines,
       colorlinks=true]{article}}

\docoptions

% Make text searchable (avoid ligatures in words like affinity)
% by loading the 'cmap' package.
\usepackage{cmap}

% Choose a font that is available as T1, specify encoding
% and load symbol definitions.
\usepackage[english]{babel}
\usepackage{lmodern}
\usepackage[T1]{fontenc}
\usepackage{textcomp}

\usepackage[noindentafter]{titlesec}
\usepackage{xifthen}
\usepackage[all]{xy}
\usepackage{html}

\usepackage{xspace}

\newboolean{draft}
\newcommand{\ifdraftspec}[2]{\ifthenelse{\boolean{draft}}{#1}{#2}}
\draftorfinal

% The spec. file will define \myspecversion and \mydraftversion.
%
% NOTE: \specversion must evaluate to a string without
% embedded \if's due to the fragile nature of its use
% within \title and \hypersetup.  Thus, \ifdraftspec cannot be used
% inside the \newcommand definition below.
%
\ifdraftspec{
  \newcommand{\specversion}{\myspecversion~\mydraftversion}
}{
  \newcommand{\specversion}{\myspecversion}
}

\ifdraftspec{
  % DRAFT FOR NOW -- first page only
  \usepackage[firstpage]{draftwatermark}
  \SetWatermarkLightness{0.80}
}{}

\ifdraftspec{
  \newcommand{\draftnote}{\\
  \vspace{1in}
  \fbox{
  \parbox{\textwidth}{
  {\bf Draft Note:}\\
  \small
  This document is a draft and has not been ratified by the UPC consortium. 
  All contents should be considered speculative and subject to change.
  Change annotations appearing in this draft are relative to the baseline Version 1.3
  Draft 1, which is believed to be semantically identical in every detail to
  UPC language specification version 1.2 (ratified May 2005). Change
  annotations in the spec body are for reviewer convenience only and are not 
  normative, nor will they appear in the final draft. 
  \\ \\
  To learn more about planned changes or participate in the UPC specification
  revision process, please visit: 
  {\tt http://code.google.com/p/upc-specification/}
  }} }
}{
  \newcommand{\draftnote}{}
}

% stabilize the margins for twoside
\setlength{\oddsidemargin}{50pt}  
\setlength{\evensidemargin}{50pt}  

% fix-up some poor vertical spacing
\usepackage{parskip}
\renewcommand{\pagebreak}{\newpage}

% Headers and footers, similar to ISO C99 spec
\usepackage{fancyhdr}
\usepackage{extramarks}
\pagestyle{fancy}
\fancyhead{}
\fancyfoot{}
\setlength{\headheight}{14.5pt}  
\renewcommand{\sectionmark}[1]{\markboth{\thesection}{#1}}
\renewcommand{\subsectionmark}[1]{\markboth{\thesubsection}{#1}}
\renewcommand{\subsubsectionmark}[1]{\markboth{\thesubsubsection}{#1}}
\renewcommand{\paragraphmark}[1]{\markboth{\theparagraph}{#1}}
\newcommand{\footersection}{ 
   \ifthenelse{\equal{\thesection}{0} 
           \OR \equal{\firstleftmark}{0} 
           \OR \equal{\firstleftmark}{INDEX} 
	       } {} %  omit empty section numbers in front-matter
       {\S\firstleftmark}}
\cfoot{\parbox{0.75\textwidth}{\centering\nouppercase{\firstrightmark}}}
\fancyfoot[LO,RE]{\footersection}
\fancyfoot[RO,LE]{\thepage}
\lhead{\mytitle}
\rhead{\myversion}

\usepackage{makeidx}

\setcounter{secnumdepth}{4}
\setcounter{tocdepth}{3}

% Make paragraphs behave like subsubsections
% See: http://www.charlietanksley.net/philtex/\
%      modifying-section-commands-with-the-titlesec-package/
\titleformat{\paragraph}[hang]{\bfseries}{\theparagraph}{1em}{}
\titlespacing*{\paragraph}{0pt}{1.25ex plus 0.5ex minus 0.25 ex}{0.5ex}

\newcounter{parnum}
\makeatletter
\@addtoreset{parnum}{section}
\@addtoreset{parnum}{subsection}
\@addtoreset{parnum}{subsubsection}
\@addtoreset{parnum}{paragraph}
\makeatother

% unnumbered section that will play nicely with toc, hyperref and footers
\newcommand{\sectionstar}[1]{
\phantomsection
\section*{#1}
\addcontentsline{toc}{section}{#1}
\markboth{0}{#1}
}

\newcommand{\doloc}{
  \ifdraftspec{
    % hack: make sure no pages after the title revert to plain style
    % (ie \makeindex)
    \renewcommand{\thispagestyle}[1]{}
    \newpage
    \setlength{\parskip}{0ex}
    \phantomsection
    \markboth{0}{List of Changes}
    \addcontentsline{toc}{section}{List of Changes}
    \renewcommand\listtablename{List of Changes}
    \hypersetup{linktocpage=true}
    \listoftables
    %\listofchanges
    \setlength{\parskip}{1.3ex}
    \newpage
  }{}
}
\newcommand{\locentry}[1]{%
  \ifdraftspec{%
    {\addcontentsline{lot}{table}{\footnotesize{#1}}}%
  }{}%
}

\newcommand{\dotoc}{
% hack: make sure no pages after the title revert to plain style
% (ie \makeindex)
\renewcommand{\thispagestyle}[1]{}
\newpage
\setlength{\parskip}{0ex}
\phantomsection
\addcontentsline{toc}{section}{Contents}
\hypersetup{linktocpage=false}
\tableofcontents
\setlength{\parskip}{1.3ex}
\newpage
}

\newcommand{\doindex}{
\newpage
% ensure correct hyperref
\phantomsection
% Hack: suppress the bogus section mark inserted by makeindex
\renewcommand{\firstleftmark}{0}
\addcontentsline{toc}{section}{Index}
\printindex
\newpage
}

% Not really a tab, but something like it for HTML
% output.  If we want real tabs, we'll probably need to use CSS.
\newcommand{\tab}{\texttt{~~~~~~}}

\newcommand{\np}{%
\addtocounter{parnum}{1}%
\latex{\hspace{-2em}\makebox[2em][l]{\arabic{parnum}}}%
\html{{\bf{\arabic{parnum}}}\tab}}

\newcommand{\npf}{\setcounter{parnum}{0}\np}

\newcommand{\sterm}[1]{\subsection{\html{#1}}
\npf \latex{{\bf #1}\\}
\index{#1}}

\newcommand{\ssterm}[1]{\subsubsection{\html{#1}}
\npf \latex{{\bf #1}\\}
\index{#1}}

\newcommand{\pterm}[1]{\paragraph{\html{#1}}
\npf \latex{{\bf #1}\\}
\index{#1}}

\setlength{\parindent}{0pt}
\setlength{\parskip}{1.3ex}

%%%%%%%%%%%%%%%%%%%%%%%%%%%%%%%%%%%%%%%%%%%%%%%%%%%%%%%%%%%%
%
% change/revision mark up.
%
% NOTE: the 'changes' annotations (\added, \changed, \deleted)
% cannot span paragraphs.
%
% fixme can be used to add 'todo' notations
\usepackage{fixme}

\ifdraftspec{
    % We extend the changes markup to include change bars.
    % The 'ncb' forms of the commands should be used in 'fragile'
    % contexts such as caption titles.
    \usepackage{changes}
    \usepackage[outerbars]{changebar}
    \newcommand{\upcspecissue}[1]{%
    \ifcase#1\relax
        \empty
      \or % issue #1
        Typesetting of section 3
      \or % issue #2
        Layout specifier '[*]' interpretation in typedef
      \or % issue #3
        Is a pointer to an array-of-shared a pointer-to-shared or a pointer-to-local?
      \or % issue #4
        Progress guarantee of upc\_notify and upc\_wait
      \or % issue #5
        Equivalence of upc\_barrier and upc\_notify plus upc\_wait
      \or % issue #6
        Relaxing the relaxed consistency same-address exception
      \or % issue #7
        Library: Atomic Memory Operations (AMO)
      \or % issue #8
        Improve UPC data layout options
      \or % issue #9
        Library: High-Performance Wall-Clock Timers (upc\_tick\_t)
      \or % issue #10
       	Add upc\_types.h to define common library types
      \or % issue #11
        UPC pointers-to-shared arithmetics
      \or % issue #12
        Library: Collective Deallocation Functions
      \or % issue #13
        Constrain sizeof a UPC shared type or object to the range of size\_t
      \or % issue #14
        Add language to the UPC spec. to further explain/constrain upc\_localsizeof semantics
      \or % issue #15
        Add constraint: a declaration of an array with indefinite block size must have compile-time constant dimensions
      \or % issue #16
        Add description of semantics of sizeof applied to a UPC shared type or expression in a static THREADS environment
      \or % issue #17
        Require that at least one dimension of a shared array type or declaration is an integral multiple of the block size
      \or % issue #18
        Clarification: UPC library functions can be implemented as macros, but must also have external linkage
      \or % issue #19
        Implement upc\_threadof, upc\_phaseof, and upc\_addrfield as operators
      \or % issue #20
        Drop strict/relaxed qualifiers when checking type compatibility with generic pointers-to-shared.
      \or % issue #21
        Define UPC implementation limits and required minimum values for same
      \or % issue #22
        Do not specify UPC support for UPC threads implemented as OS threads
      \or % issue #23
        Do not specify UPC support for UPC threads implemented as OS threads
      \or % issue #24
        Define GASP (Global Address Space Performance monitoring) support as an optional UPC library
      \or % issue #25
        Provide additional examples to illustrate aspects that are difficult to understand
      \or % issue #26
        limit allowed range of values for barrier id values
      \or % issue #27
        Clarification: "long double" and "long double complex" support
      \or % issue #28
        stdio behavior
      \or % issue #29
        Correct example 2 in the UPC specification in the discussion of upc\_forall
      \or % issue #30
        Deprecate support for static THREADS compilation
      \or % issue #31
        block size of [*] should be capped at UPC\_MAX\_BLOCK\_SIZE
      \or % issue #32
        modification: THREADS/MYTHREAD have "integral value" rather than "type int"
      \or % issue #33
        clarification: MYTHREAD and THREADS are expressions (cannot assign to or take address of them)
      \or % issue #34
        Define UPC\_OFF\_MAX and required minumum value
      \or % issue #35
        Define accesses to UPC shared bit fields as atomic with respect to accesses to adjacent fields.
      \or % issue #36
        Add language to clarify shared/non-shared alias relationships
      \or % issue #37
        Library: Shared-pointer privatizability (castability) functions
      \or % issue #38
        Deprecate the '[*]' layout qualifier
      \or % issue #39
        Can layout qualifiers applied to shared scalars be eliminated?
      \or % issue #40
        Language Extension:  generalize layout qualifiers so that they behave like C99 dynamic array types (VLA's).
      \or % issue #41
        Library: non-blocking memory copy extensions
      \or % issue #42
        Library: collectives 2.0
      \or % issue #43
        Can pointer-to-shared arithmetic be specified in terms of the / and \% operators of the C language?
      \or % issue #44
        Can pointer-to-shared arithmetic be specified in terms of the / and \% operators of the C language?
      \or % issue #45
        Program exit code: advice to implementers
      \or % issue #46
        Request for a plain English description of memory consistency + a portable battery of consistency tests
      \or % issue #47
        upc\_forall semantics: upc affinity expression and side-effects
      \or % issue #48
        UPC progress guarantees
      \or % issue #49
        clarification: unlock of freed lock
      \or % issue #50
        clarification: overlapping memory copies undefined by presence of "restrict" keyword
      \or % issue #51
        clarification: revise text to eliminate an ambiguity in barrier matching semantics
      \or % issue #52
        Library: topology query (node awareness)
      \or % issue #53
        Consider allowing static allocation of shared data whose affinity starts on a non-zero thread
      \or % issue #54
        Write section "Proposed Additions and Extensions"
      \or % issue #55
        Miscellaneous specification document typographical errors
      \or % issue #56
        Clarification: Inclusion of <upc.h> is not required for strictly conforming UPC programs
      \or % issue #57
        upc\_lock\_free: the effect of free-ing a UPC lock held by another thread is undefined
      \or % issue #58
        Bring Conformance section into agreement with the C99 specification
      \or % issue #59
        The affinity test on integer values in a upc\_forall statement is undefined for negative values
      \or % issue #60
        Clarification: the use of a non- shared qualified operand of upc\_blocksize (and related operators) is erroneous
      \or % issue #61
        Clarification: data tearing and read/write ordering
      \or % issue #62
        Add an "Informative Warnings" Annex
      \or % issue #63
        Provide a "UPC Implementer's Guide" document
      \or % issue #64
        Barrier statement optional expression type is an integer type, not just 'int'
      \or % issue #65
        Define a null pointer-to-shared in terms of C99's "null pointer constant"
      \or % issue #66
        Clarification: each collection of related library functions should have its own associated header file
      \or % issue #67
        Add a kitchen sink upc\_everything.h header file?
      \or % issue #68
        Clarification: the re-specification of a same-valued layout qualifer is allowed
      \or % issue #69
        Clarification: how is an empty layout qualfier, or a qualifier of [1] handled when present in a series of qualifiers, or via a typedef chain?
      \or % issue #70
        Clarification: how is type compatibility defined when one/both pointer-to-shared target types are incomplete?
      \or % issue #71
        Clarification: can the [*] layout qualifier be applied to a typedef?
      \or % issue #72
        Clarification: are layout qualifiers allowed in shared scalar declarations?
      \or % issue #73
        Proposal: pointer-to-shared comparion should ignore phase
      \or % issue #74
        Clarification: cast of (shared <type> * shared) is an error
      \or % issue #75
        Clarification: a cast from non-shared to shared is an error
      \or % issue #76
        Enhancement: fully anonymous barrier
      \or % issue #77
        disallow & applied to shared bitfields
      \or % issue #78
        Remove shared array initialization
      \or % issue #79
        should not hardcode path of makeindex
      \or % issue #80
        Library for signalling put and point-to-point synchronization (upc\_sem\_t)
      \or % issue #81
        Unintentional wording regressions relative to 1.2
      \or % issue #82
        Remove the deprecated upc\_local\_alloc function
      \or % issue #83
        Strengthen the "default" pragma from "implementation-defined" to "relaxed"
      \or % issue #84
        Release Spec 1.3 Draft 1 for public consumption
      \or % issue #85
        Fix ambiguities and inconsistencies involving multi-dimensional shared arrays
      \or % issue #86
        Adopt C11 as base language
      \or % issue #87
        Publish Draft 2
      \or % issue #88
        Nested upc\_forall semantics
      \or % issue #89
        Publicity materials for SC12
      \or % issue #90
        limit allowed range of values for program exit (via exit or upc\_global\_exit)
      \or % issue #91
        Library section boilerplate spec text
      \or % issue #92
        Collective support for long double and long double complex
      \or % issue #93
        Enhancement: Library for PTS arithmetic with parametric blocksize
      \or % issue #94
        Clarification: dynamic threads and the constraint "when multiplied by an integer expression"
      \or % issue #95
        Clarification: can THREADS appeareclarator in more than once in a PTS typedef (in a dynamic threads environment)?
    \else
        \empty
    \fi
}

    \definechangesauthor[name={Dan Bonachea}]{DB}
    \definechangesauthor[name={Bill Carlson}]{WC}
    \definechangesauthor[name={Gary Funck}]{GF}
    \definechangesauthor[name={Nick Park}]{NP}
    \definechangesauthor[name={Steven Vormwald}]{SV}
    \definechangesauthor[name={Yili Zheng}]{YZ}
    \setlength{\changebarwidth}{2pt}
    \setcounter{changebargrey}{0}
    \setauthormarkup{}
    \setauthormarkuptext{name}
    \setaddedmarkup{\textcolor{blue}{#1}}
    \setdeletedmarkup{\textcolor{red}{\sout{#1}}}
    \newcommand{\authorid}[1]{}
    \newcommand{\specissueref}[1]{%
      \ifthenelse{\isempty{#1}}%
	{}%
        {%
	  \def\issuesummary{\upcspecissue{#1}}%
	  \ifthenelse{\isempty{\issuesummary}}%
	    {Issue \##1}%
	    {%
               \href{http://code.google.com/p/upc-specification/issues/%
detail?id=#1}{Issue \##1: \issuesummary}%
	    }%
	}%
    }
    % \xchangenote is an inline change note that gets compiled away in final version
    \newcommand{\xchangenote}[3][]{%
      \cbstart%
      \xaddedncb[{#1}]{#2}{#3}%
      \cbend}
    \newcommand{\xadded}[3][]{%
      \cbstart%
      \xaddedncb[{#1}]{#2}{#3}%
      \cbend}
    \newcommand{\xdeleted}[3][]{%
      \cbstart%
      \xdeletedncb[{#1}]{#2}{#3}%
      \cbend}
    \newcommand{\xreplaced}[4][]{%
      \cbstart%
      \xreplacedncb[{#1}]{#2}{#3}{#4}%
      \cbend}

    % use a separate footnote numbering space for change annotations
    % to avoid disturbing the footnote numbering of the original document
    \newcounter{changefootnote}
    \newcounter{savefootnote}
    \newcommand{\openchangefootnotes}{%
      \setcounter{savefootnote}{\value{footnote}}% Swap footnote counter 
      \setcounter{footnote}{\value{changefootnote}}%
      \renewcommand{\thefootnote}{\roman{footnote}}% Modify footnote printing
    }
    \newcommand{\closechangefootnotes}{%
      \setcounter{changefootnote}{\value{footnote}}% Swap footnote counter 
      \setcounter{footnote}{\value{savefootnote}}%
      \renewcommand{\thefootnote}{\arabic{footnote}}% Restore footnote printing
    }
    % use \truefootnote for document footnotes that appear inside a change annotation
    \newcommand{\truefootnote}[1]{%
      \closechangefootnotes%
      \footnote{#1}%
      \openchangefootnotes%
    }%

    % The following 'FN' versions of the commands are for annotating changes
    % inside footnotes, where the changes package breaks because it tries to 
    % insert a nested footnote, which is not supported in LaTeX.
    % These are two-part commands: 
    % \xaddedFN and friends appear inline and just take the textual change as arguments
    % \xFNinfo must appear outside the footnote and take the author/issue info to output the change footnote
    \let\xfootnoteorig\footnote
    \newcommand{\discard}[1]{}
    \newcommand{\xaddedFN}[1]{%
      \cbstart%
      \openchangefootnotes%
      \let\footnote\discard%
      \added[]{#1}\footnotemark%
      \let\footnote\xfootnoteorig%
      \closechangefootnotes%
      \cbend%
    }
    \newcommand{\xdeletedFN}[1]{%
      \cbstart%
      \openchangefootnotes%
      \let\footnote\discard%
      \deleted[]{#1}\footnotemark%
      \let\footnote\xfootnoteorig%
      \closechangefootnotes%
      \cbend%
    }
    \newcommand{\xreplacedFN}[2]{%
      \cbstart%
      \openchangefootnotes%
      \let\footnote\discard%
      \replaced[]{#1}{#2}\footnotemark%
      \let\footnote\xfootnoteorig%
      \closechangefootnotes%
      \cbend%
    }
    \newcommand{\xFNinfo}[2][]{%
      \openchangefootnotes%
      \ifthenelse{\isempty{#2}}%
        {\footnotetext{#1}}%
        {\locentry{\protect\specissueref{#2}}%
	 \footnotetext{\authorid{#1: }\specissueref{#2}}}%
      \closechangefootnotes%
    }

    % The following 'ncb' forms of the commands are for use
    % in 'fragile' environments (such as figure captions)
    % where change bars do not work.
    \newcommand{\xaddedncb}[3][]{%
      \openchangefootnotes%
      \ifthenelse{\isempty{#2}}%
        {\added[\authorid{#1}]{#3}}%
	{\locentry{\protect\specissueref{#2}}%
	 \added[remark={\protect\specissueref{#2}},\authorid{#1}]{#3}}%
      \closechangefootnotes%
    }
    \newcommand{\xdeletedncb}[3][]{%
      \openchangefootnotes%
      \ifthenelse{\isempty{#2}}%
        {\deleted[\authorid{#1}]{#3}}%
	{\locentry{\protect\specissueref{#2}}%
	 \deleted[remark={\protect\specissueref{#2}},\authorid{#1}]{#3}}%
      \closechangefootnotes%
    }
    \newcommand{\xreplacedncb}[4][]{%
      \openchangefootnotes%
      \ifthenelse{\isempty{#2}}%
        {\replaced[\authorid{#1}]{#3}{#4}}%
	{\locentry{\protect\specissueref{#2}}%
	 \replaced[remark={\protect\specissueref{#2}},\authorid{#1}]{#3}{#4}}%
      \closechangefootnotes%
    }
  }
  {
    \newcommand{\xchangenote}[3][]{}
    \newcommand{\xadded}[3][]{#3}
    \newcommand{\xdeleted}[3][]{}
    \newcommand{\xreplaced}[4][]{#3}
    \newcommand{\xaddedFN}[1][]{#1}
    \newcommand{\xdeletedFN}[1][]{}
    \newcommand{\xreplacedFN}[2][]{#1}
    \newcommand{\xFNinfo}[2][]{}
    \newcommand{\xaddedncb}[3][]{#3}
    \newcommand{\xdeletedncb}[3][]{}
    \newcommand{\xreplacedncb}[4][]{#3}
    \let\truefootnote\footnote
    \newenvironment{changebar}{}{}
    \newcommand{\cbstart}{}
    \newcommand{\cbend}{}
  }

% Cross-document section references 
% currently these must be manually maintained
\def \upcopsection   {UPC Language Specification, Section 7.3.1\xspace}
\def \upctypesection {UPC Language Specification, Section 7.3.2\xspace}
\def \upcflagsection {UPC Language Specification, Section 7.3.3\xspace}

% The special comments below are processed by latex2html.
% Everything between them will be skipped by latex2html.
%begin{latexonly}
  \usepackage{ifpdf}
  \ifpdf
    \usepackage{pdfsync}
    \usepackage{hyperref}
    \hypersetup{bookmarksdepth=4,
		% set the title in the pdf file properties
                pdftitle={\mytitle,\ \myversion},
		pdfauthor={UPC Consortium},
                pageanchor=true,
		% default view on open
		pdfpagemode=UseOutlines,
		% view after clicking a link
		pdfview=Fit
		}
  \fi
%end{latexonly}
 
% DO NOT PLACE ANY OTHER PACKAGES HERE - hyperref must come last!
